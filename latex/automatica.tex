\section{Automatica}
\subsection{Sistemi dinamici a tempo discreto}

Un sistema dinamico a tempo discreto è il modello matematico di un oggetto che interagisce con l’ambiente circostante attraverso canali di ingresso e di uscita rappresentati attraverso vettori $u$ e $y$ di variabili dipendenti dal tempo. Si differenziano dalla classe dei sistemi a tempo continuo dal fatto che in questo caso il tempo è rappresentato come una variabile intera $k \in \mathbb{Z}$.

Si avrà pertanto che in ogni istante di tempo $k$ il sistema modificherà le proprie uscite sulla base dei segnali in ingresso.

Il vettore $u \in \mathbb{R}^m$ rappresenta i segnali che l’oggetto riceve in ingresso dall’esterno mentre il vettore $y \in \mathbb{R}^p$ rappresenta i segnali che l’oggetto fornisce in uscita.

In generale il comportamento del sistema non dipende esclusivamente da questi due vettori, ovvero non vi è un legame diretto tra ingresso e uscita: infatti il sistema può avere uno stato interno che evolve in funzione degli ingressi e degli stati precedenti. Lo stato di un sistema è rappresentato da un vettore $x \in \mathbb{R}^n$.

Il modello del sistema è pertanto costituito da equazioni che descrivono l’evoluzione dello stato del sistema in funzione dell’ingresso, dello stato e del tempo ed esprimono la relazione d'uscita:
\begin{subequations}
\begin{align}
x_{k+1} &= f(x_k,u_k,k) \\
y_k &= g(x_k,u_k,k)
\end{align}
\end{subequations}
dove $f$ e $g$ sono opportune funzioni vettoriali.



Consideriamo una particolare classe di sistemi, quelli lineari strettamente propri, in cui $f$ e $g$ sono funzioni lineari e l’uscita non dipende direttamente dall’ingresso ma solo dallo stato. In questo caso le equazioni generali del sistema sono:
\begin{subequations}
\begin{align}
x_{k+1} &= A_kx_k + B_ku_k \\
y_k &= C_kx_k
\end{align}
\end{subequations}

dove $A,B,C$ sono matrici di coefficienti in generale variabili nel tempo. Se tali matrici sono costanti al variare di $k$ il sistema si dice \textit{tempo invariante}.
\newpage
\subsection{Sistemi stocastici}

Il modello matematico di un sistema è un'astrazione che necessariamente deve trascurare alcuni fenomeni che sarebbero troppo complessi da descrivere. Nel caso in cui gli effetti di tali fenomeni non siano trascurabili, è possibile considerarli nel modello rappresentandoli come fenomeni stocastici, ovvero come variabili aleatorie.

Tali variabili possono anche modellizzare le incertezze nella misura delle uscite del processo, ad esempio nel caso in cui esse siano affette da rumore.

Il modello del sistema tenendo conto di tali fenomeni può essere riscritto come:
\begin{subequations}
\label{rumlinsys}
\begin{align}\label{rumlinsys1}
x_{k+1} &= A_kx_k + B_ku_k + W_kw_k \\
\label{rumlinsys2}
y_k &= C_kx_k + v_k
\end{align}
\end{subequations}

\noindent Le ipotesi che facciamo per caratterizzare i termini $w$ e $v$ sono:
\begin{itemize}
\item $w$ e $v$ sono vettori casuali gaussiani a media nulla:
\begin{subequations}
\label{inizioipotesistatistiche}
\begin{align}
 w_k &\sim \mathcal{N}(0,\,Q_k), \qquad Q_k=Q_k^T > 0\\
 v_k &\sim \mathcal{N}(0,\,R_k), \qquad R_k=R_k^T > 0
\end{align}
\end{subequations}
\item $w$ e $v$ sono incorrelati:
\begin{equation}
 E[w_{k_1}v^T_{k_2}]=0\qquad \forall\, k_1,k_2 \geq 0\
\end{equation}
\item $w$ e $v$ sono bianchi:
\begin{subequations}
\begin{align}
 E[w_{k_1}w^T_{k_2}]&=0\qquad \forall\, k_1 \neq k_2 \\
 E[v_{k_1}v^T_{k_2}]&=0\qquad \forall\, k_1 \neq k_2
\end{align}
\end{subequations}
\item $x_0$ è un vettore casuale gaussiano con media e covarianza note:
\begin{equation}\label{key}
 x_0 \sim \mathcal{N}(\bar{x}_0,\,P_0), \qquad P_0=P_0^T > 0
\end{equation}
\item $w$ e $v$ sono incorrelati con $x_0$:
\begin{subequations}
\label{fineipotesistatistiche}
\begin{align}
 E[x_0w^T_k]&=0\qquad \forall\, k \geq 0 \\
 E[x_0v^T_k]&=0\qquad \forall\, k \geq 0
\end{align}
\end{subequations}
\end{itemize}

Con le ipotesi fatte è possibile passare al problema della progettazione di un osservatore che restituisca una stima dello stato interno del sistema filtrando i rumori e le incertezze sull'evoluzione dello stato e sulla misura dell'uscita.
\newpage