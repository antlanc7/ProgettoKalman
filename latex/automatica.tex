\section{Automatica}
\subsection{Sistemi dinamici a tempo discreto}

Un sistema dinamico a tempo discreto è il modello matematico di un oggetto che interagisce con l’ambiente circostante attraverso canali di ingresso e di uscita che sono rappresentati attraverso vettori, $\vett{u}$ e $\vett{y}$, di variabili dipendenti dal tempo. La differenza principale dai sistemi a tempo continuo è che in questo caso il tempo è rappresentato come una variabile intera $k$.

Si avrà pertanto che per ogni istante di tempo $k$ il sistema riceverà dei segnali in ingresso e risponderà con dei segnali in uscita.

Il vettore $\vett{u} \in \mathbb{R}^m$ rappresenta i segnali che l’oggetto riceve dall’esterno mentre il vettore $\vett{u} \in \mathbb{R}^p$ rappresenta i segnali che l’oggetto dà in uscita.

In generale il comportamento del sistema non dipende esclusivamente da questi due vettori, ovvero non vi è un legame diretto tra ingresso e uscita: infatti il sistema ha uno stato interno che evolve in funzione degli ingressi e degli stati precedenti. In particolare lo stato di un sistema può essere a sua volta rappresentato da un vettore $\vett{x} \in \mathbb{R}^n$.

Il modello del sistema è pertanto costituito da equazioni che descrivono l’evoluzione dello stato del sistema in funzione dell’ingresso, dello stato e del tempo e esprimono la relazione d'uscita:
\begin{align*}
\vett{x}_{k+1} &= \vett{f}(\vett{x}_k,\vett{u}_k,k) \\
\vett{y}_k &= \vett{g}(\vett{x}_k,\vett{u}_k,k)
\end{align*}
dove $\vett{f}$ e $\vett{g}$ sono opportune funzioni vettoriali.

\subsection{Sistemi lineari stocastici}
Consideriamo una particolare tipologia di sistemi, quelli lineari strettamente propri, in cui le funzioni $\vett{f}$ e $\vett{g}$ sono appunto funzioni lineari e l’uscita non dipende esplicitamente dall’ingresso. In questo caso le equazioni si possono genericamente scrivere come:
\begin{align*}
\vett{x}_{k+1} &= \matr{A}_k\vett{x}_k + \matr{B}_k\vett{u}_k \\
\vett{y}_k &= \matr{C}_k\vett{x}_k
\end{align*}
dove $\matr{A},\matr{B},\matr{C}$ sono matrici di coefficienti variabili nel tempo.

Tuttavia tali modelli sono approssimazioni ideali che possono essere valide in alcuni contesti, mentre in altri è necessario tener conto delle incertezze e delle imprecisioni che si hanno nella misura dei segnali di ingresso e di uscita del sistema.

Il modello del sistema tenendo conto di queste incertezze può essere riscritto come:
\begin{align*}
\vett{x}_{k+1} &= \matr{A}_k\vett{x}_k + \matr{B}_k\vett{u}_k + \vett{v}_k \\
\vett{y}_k &= \matr{C}_k\vett{x}_k + \vett{w}_k
\end{align*}

\noindent Le ipotesi che possiamo fare per caratterizzare i termini $\vett{v}$ e $\vett{w}$ sono:
\begin{itemize}
\item $\vett{v}$ e $\vett{w}$ sono vettori casuali gaussiani a media nulla: 
\begin{align*}
 \vett{v}_k &\sim \mathcal{N}(0,\,\matr{Q}_k) \\
 \vett{w}_k &\sim \mathcal{N}(0,\,\matr{R}_k)
\end{align*}
\item $\vett{v}$ e $\vett{w}$ sono incorrelati:
 \[E[\vett{v}_{k_1}\vett{w}^T_{k_2}]=0\qquad \forall\, k_1,k_2 \geq 0\]
\item $\vett{v}$ e $\vett{w}$ sono bianchi: 
\begin{align*}
 E[\vett{v}_{k_1}\vett{v}^T_{k_2}]&=0\qquad \forall\, k_1 \neq k_2 \\
 E[\vett{w}_{k_1}\vett{w}^T_{k_2}]&=0\qquad \forall\, k_1 \neq k_2
\end{align*}
\item $\vett{x_0}$ è un vettore casuale gaussiano con media e varianza note:
\[\vett{x_0} \sim \mathcal{N}(\vett{\bar{x}_0},\,\matr{P_0})\]
\item $\vett{v}$ e $\vett{w}$ sono incorrelati con $\vett{x_0}$:
\begin{align*}
 E[\vett{x_0}\vett{v}^T_k]&=0\qquad \forall\, k \geq 0 \\
 E[\vett{x_0}\vett{w}^T_k]&=0\qquad \forall\, k \geq 0
\end{align*}
\end{itemize}

Con le ipotesi fatte è possibile passare al problema della progettazione di un osservatore che restituisca una stima dello stato interno del sistema filtrando i rumori e le incertezze sull'evoluzione dello stato e sulla misura dell'uscita.