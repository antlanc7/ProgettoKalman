\documentclass[12pt,a4paper]{article}
\usepackage[utf8]{inputenc}
\usepackage[italian]{babel}
\usepackage{amsmath}
\usepackage{amsfonts}
\usepackage{amssymb}

\newcommand*{\matr}{\mathbf}
\newcommand*{\vett}{\mathbf}

\title{Automatica}

\begin{document}

\section{Automatica}
\subsection{Sistemi dinamici a tempo discreto}

Un sistema dinamico a tempo discreto è il modello matematico di un oggetto che interagisce con l’ambiente circostante attraverso canali di ingresso e di uscita che sono rappresentati attraverso vettori, $\vett{u}$ e $\vett{y}$, di variabili dipendenti dal tempo. La differenza principale dai sistemi a tempo continuo è che in questo caso il tempo è rappresentato come una variabile intera $k$.

Si avrà pertanto che per ogni istante di tempo $k$ il sistema riceverà dei segnali in ingresso e risponderà con dei segnali in uscita.

Il vettore $\vett{u} \in \mathbb{R}^m$ rappresenta i segnali che l’oggetto riceve dall’esterno mentre il vettore $\vett{u} \in \mathbb{R}^p$ rappresenta i segnali che l’oggetto dà in uscita.

In generale il comportamento del sistema non dipende esclusivamente da questi due vettori, ovvero non vi è un legame diretto tra ingresso e uscita: infatti il sistema ha uno stato interno che evolve in funzione degli ingressi e degli stati precedenti. In particolare lo stato di un sistema può essere a sua volta rappresentato da un vettore $\vett{x} \in \mathbb{R}^n$.

Il modello del sistema è pertanto costituito da equazioni che descrivono l’evoluzione dello stato del sistema in funzione dell’ingresso, dello stato e del tempo e esprimono la relazione d'uscita:
\begin{align*}
\vett{x}(k+1) &= \vett{f}(\vett{x}(k),\vett{u}(k),k) \\
\vett{y}(k) &= \vett{g}(\vett{x}(k),\vett{u}(k),k)
\end{align*}
dove $\vett{f}$ e $\vett{g}$ sono opportune funzioni vettoriali.

\subsection{Sistemi lineari stocastici}
Consideriamo una particolare tipologia di sistemi, quelli lineari strettamente propri, in cui le funzioni $\vett{f}$ e $\vett{g}$ sono appunto funzioni lineari e l’uscita non dipende esplicitamente dall’ingresso. In questo caso le equazioni si possono genericamente scrivere come:
\begin{align*}
\vett{x}(k+1) &= \matr{A}(k)\vett{x}(k) + \matr{B}(k)\vett{u}(k) \\
\vett{y}(k) &= \matr{C}(k)\vett{x}(k)
\end{align*}
dove $\matr{A},\matr{B},\matr{C}$ sono matrici di coefficienti variabili nel tempo.

Tuttavia tali modelli sono approssimazioni ideali che possono essere valide in alcuni contesti, mentre in altri è necessario tener conto delle incertezze e delle imprecisioni che si hanno nella misura dei segnali di ingresso e di uscita del sistema.

Tali incertezze possono essere modellizzate come vettori casuali:
\begin{align*}
\vett{x}(k+1) &= \matr{A}(k)\vett{x}(k) + \matr{B}(k)\vett{u}(k) + \vett{v}(k) \\
\vett{y}(k) &= \matr{C}(k)\vett{x}(k) + \vett{w}(k)
\end{align*}
In particolare si possono fare alcune ipotesi su tali variabili casuali:

\end{document}