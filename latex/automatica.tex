\documentclass[12pt,a4paper]{article}
\usepackage[utf8]{inputenc}
\usepackage[italian]{babel}
\usepackage{amsmath}
\usepackage{amsfonts}
\usepackage{amssymb}
\title{Automatica}

\begin{document}

\section{Automatica}
\subsection{Sistemi dinamici a tempo discreto}

Un sistema dinamico a tempo discreto è il modello matematico di un oggetto che interagisce con l’ambiente circostante attraverso canali di ingresso e di uscita che sono rappresentati attraverso vettori, $\mathbf{u}$ e $\mathbf{y}$, di variabili dipendenti dal tempo. La differenza principale dai sistemi a tempo continuo è che in questo caso il tempo è rappresentato come una variabile intera $k$.

Si avrà pertanto che per ogni istante di tempo $k$ il sistema riceverà dei segnali in ingresso e risponderà con dei segnali in uscita.

Il vettore $\mathbf{u} \in \mathbb{R}^m$ rappresenta i segnali che l’oggetto riceve dall’esterno mentre il vettore $\mathbf{u} \in \mathbb{R}^p$ rappresenta i segnali che l’oggetto dà in uscita.


In generale il comportamento del sistema non dipende esclusivamente da questi due vettori, ovvero non vi è un legame diretto tra ingresso e uscita: infatti il sistema ha uno stato interno che evolve in funzione degli ingressi e degli stati precedenti. In particolare lo stato di un sistema può essere a sua volta rappresentato da un vettore $\mathbf{x} \in \mathbb{R}^n$.


Il modello del sistema è pertanto costituito da equazioni che descrivono l’evoluzione dello stato del sistema in funzione dell’ingresso, dello stato e del tempo e esprimono la relazione d'uscita:
\[
\begin{cases}
\mathbf{x}(k+1) = \mathbf{f}(\mathbf{x}(k),\mathbf{u}(k),k) \\
\mathbf{y}(k) = \mathbf{g}(\mathbf{x}(k),\mathbf{u}(k),k)
\end{cases}
\]
dove f e g sono opportune funzioni vettoriali.

Consideriamo una particolare tipologia di sistemi, quelli lineari strettamente propri, in cui le funzioni f e g sono appunto funzioni lineari e l’uscita non dipende esplicitamente dall’ingresso. In questo caso le equazioni si possono genericamente scrivere come:
\[
\begin{cases}
\mathbf{x}(k+1) = A(k)\mathbf{x}(k) + B(k)\mathbf{u}(k) \\
\mathbf{y}(k) = C(k)\mathbf{x}(k)
\end{cases}
\]
dove A,B,C sono matrici di coefficienti variabili nel tempo.

Tuttavia tali modelli sono approssimazioni ideali che possono essere valide in alcuni contesti, mentre in altri è necessario tener conto delle incertezze e delle imprecisioni che si hanno nella misura dei segnali di ingresso e di uscita del sistema.

Tali incertezze possono essere modellizzate come vettori casuali:
\[
\begin{cases}
x(k+1) = A(k)x(k) + B(k)u(k) + v(k)
y(k) = C(k)x(k) + w(k)
\end{cases}
\]
In particolare si possono fare alcune ipotesi su tali variabili casuali:

\end{document}