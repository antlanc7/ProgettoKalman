\documentclass{article}
\usepackage[utf8]{inputenc}
\usepackage[italian]{babel}
\usepackage{amsmath}
\usepackage{amsfonts}
\usepackage{amssymb}
\usepackage{url}

\usepackage{hyperref}
\hypersetup{
     colorlinks = true,
     linkcolor = black,
     urlcolor = blue
     }

\usepackage{natbib}
\usepackage{graphicx}
\usepackage{caption}
\usepackage{wrapfig}
\graphicspath{ {./immagini/} }
\captionsetup[figure]{labelformat=empty}

\usepackage{xcolor}
\usepackage{listings}
\lstdefinestyle{customc}{
  belowcaptionskip=1\baselineskip,
  breaklines=true,
  frame=L,
  xleftmargin=\parindent,
  language=Matlab,
  showstringspaces=false,
  basicstyle=\footnotesize\ttfamily,
  keywordstyle=\bfseries\color{green!40!black},
  commentstyle=\itshape\color{orange!40!black},
  identifierstyle=\color{blue},
  stringstyle=\color{orange}
}

\lstset{escapechar=@,style=customc,tabsize=2}

%\newcommand*{\matr}{\mathbf}
%\newcommand*{\vett}{\mathbf}

\numberwithin{equation}{section}

\title{Filtro di Kalman}
\author{Antonio Lanciotti, Lorenzo D'Agostino, Arment Pelivani}
\date{2019-2020}

\begin{document}

\maketitle

\begin{figure}[ht]
\centering
\includegraphics[scale=1]{Rudolf_Kalman} 
\caption{Rudolf E. Kalman}
\label{fig:kalman}
\end{figure}

\newpage

\tableofcontents

\newpage



\section{Introduzione}
Il \textit{filtro di Kalman} è un osservatore ottimo dello stato per sistemi lineari in presenza di rumori gaussiani.

La sua versatilità ed utilità lo ha portato ad innumerevoli applicazioni quali il controllo di veicoli di ogni genere (aerospaziali, navali ...)\cite{veicoli} , robotica e object tracking \cite{Vision}, ricostruzione di segnali affetti da disturbi, stima dello State Of Charge \cite{SOC} delle batterie e molto altro. Trova anche spazio in applicazioni finanziarie\cite{Finanza}.
In questa relazione vengono presentati i procedimenti matematici che permettono il raggiungimento delle equazioni che lo descrivono.

Verrà poi discussa un' applicazione del filtro, inerente al problema della ricostruzione di segnali affetti da rumore, realizzata in ambiente \textit{MATLAB}.

\newpage

\section{Cenni di probabilità}

Il filtro di Kalman è un algoritmo che mira alla ricostruzione dello stato interno di un sistema basandosi unicamente su una serie di misurazioni che, a causa di limiti costruttivi, sono soggette a rumore.

A causa della natura deterministica del problema, risulta necessario affrontare alcuni aspetti della teoria della probabilità, in particolare ci soffermeremo sul concetto di variabile aleatoria normale, o Gaussiana, con l'intento di fornire un modello matematico per gli errori di misura che siamo costretti ad affrontare.

\subsection{Variabili aleatorie o casuali.}

Una variabile casuale/aleatoria è una variabile che può assumere valori diversi in dipendenza da qualche fenomeno aleatorio.
In particolare diremo che una variabile casuale $X$ si dice continua se esiste una funzione $f(x)$ definita su tutto $\mathbb{R}$ : $P(X \in B) = \int_B f(x) dx$ dove la funzione $f$ si dice \textit{densità di probabilità} della variabile casuale $X$.

Una variabile aleatoria è quindi una variabile che può assumere valori appunto casuali la cui probabilità dipende dalla funzione di densità di probabilità ad essa associata. 

Le variabili casuali quindi risultano essere un valido strumento matematico per la modellazione dei rumori.

In particolare utilizzeremo le variabili casuali cosiddette Gaussiane o normali che sono caratterizzate dalla funzione densità di probabilità: \[f(x) = \frac{1}{{\sigma \sqrt {2\pi } }}e^{-\frac{1}{2}{(\frac{x-\mu}{\sigma})}^2}\] Per poterle descrivere a pieno dobbiamo però introdurre il concetto di media/valore atteso, di varianza e di covarianza.

\subsubsection{Valore atteso}

Nella teoria della probabilità il valore atteso di una variabile casuale $X$, è un numero indicato con $E[X]$ che formalizza l'idea di valore medio di un fenomeno aleatorio.
\[\ {\mathbb  {E}}[X]=\int _{{-\infty }}^{{\infty }}xf(x)dx\]

\noindent  Si noti che l'operatore valore atteso è lineare: \[E[aX + bY] = aE[X] + bE[Y]\]




\subsubsection{Varianza}

La varianza di una variabile aleatoria è una funzione, che fornisce una misura della variabilità dei valori assunti dalla variabile stessa; nello specifico, la misura di quanto essi si discostino quadraticamente rispettivamente dal valore atteso.

La varianza della variabile aleatoria $X$ è definita come il valore atteso del quadrato della variabile aleatoria centrata $X - E[X]$ :\[Var(X) = E[(X - E[X])^2]\]

\subsubsection{Covarianza}

In statistica e in teoria della probabilità, la covarianza di due variabili aleatorie è un numero che fornisce una misura di quanto le due varino assieme, ovvero della loro dipendenza.

La covarianza di due variabili aleatorie $X$ e $Y$ è il valore atteso dei prodotti delle loro distanze dalla media: \[Cov(X,Y)= E [(X-E[X])(Y-E[Y])]\]
Due variabili casuali si dicono \textit{incorrelate} se la loro covarianza è nulla.

La covarianza può essere considerata una generalizzazione della varianza \[Var(X) = Cov(X,X)\]

\subsection{Variabili gaussiane e modellazione dei rumori.}

Le variabili gaussiane sono particolari variabili aleatorie caratterizzate da due parametri, $\mu$ e $\sigma^2$, e sono indicate tradizionalmente con: \[X \sim N(\mu ,\sigma^2)\]
Si può dimostrare che per le variabili gaussiane vale che: \[E[X]= \mu \qquad Var[X]= \sigma^2\]

Come anticipato possiamo modellizzare i vettori di disturbo del sistema che consideriamo, attraverso l'utilizzo di variabili aleatorie gaussiane a media nulla e varianza $\sigma^2$, di dimensioni conformi a quelle del sistema considerato.

\newpage
\subsection{Vettori casuali}

Un vettore casuale è un vettore i cui elementi sono essi stessi variabili casuali.

Risulta necessario estendere le definizioni date in precedenza per caratterizzare rumori che agiscono su sistemi non scalari.

\subsubsection{Valore atteso}

Si dice valore atteso del vettore casuale $x \in \mathbb{R}^n$ il vettore dei valori attesi delle variabili casuali che lo compongono: \[E[\vett{x}] = \begin{pmatrix}
E[x_1] & E[x_2] & \dots & E[x_n]
\end{pmatrix}^T
\]
Si definisce inoltre il valore quadratico medio di $\vett{x}$ come $E[\vett{x}^T \vett{x}]$.

\subsubsection{Matrice di covarianza}

Si definisce matrice di covarianza del vettore casuale $\vett{x} \in \mathbb{R}^n$ la matrice $n \times n$: \[ Cov(\vett{x}, \vett{x}) = E[(\vett{x}-E[\vett{x}])(\vett{x}-E[\vett{x}])^T]\]
Per come è definita, la matrice di covarianza è una matrice simmetrica semidefinita positiva i cui elementi $\sigma^2_{ij}$ sono le covarianze tra gli elementi $x_i$ e $x_j$ del vettore $\vett{x}$.

A sua volta si definisce la matrice di \textit{cross-covarianza} tra due vettori casuali $\vett{x}$ e $\vett{y}$, la matrice \[ Cov(\vett{x}, \vett{y}) = E[(\vett{x}-E[\vett{x}])(\vett{y}-E[\vett{y}])^T]\]
Due vettori $\vett{x}$ e $\vett{y}$ si dicono \textit{incorrelati} se $Cov(\vett{x},\vett{y}) = 0$.

\subsection{Teoria della stima}
Per stima si intende il processo di inferire il valore di una variabile casuale $X$ di interesse dall'osservazione di un'altra variabile casuale $Y$ che dipenda in qualche modo da $X$. Le variabili coinvolte sono pertanto:
\begin{itemize}
	\item la variabile da stimare $X \in \mathbb{R}^n$
	\item la variabile osservata $Y \in \mathbb{R}^p$
\end{itemize}

Si definisce \textit{stimatore di $X \in \mathbb{R}^n$ basato sull’osservazione $Y \in \mathbb{R}^p$} una funzione $g:\mathbb{R}^p \rightarrow \mathbb{R}^n$ operante sull'osservazione $Y$ per produrre una stima $\widehat{X}=g(Y)$.
Dato che le variabili in gioco sono casuali, anche la stima sarà una variabile casuale. Distinguiamo la \textit{variabile casuale stima} $\widehat{X}$ da $\hat{x}$ che è il particolare valore della stima ottenuto dall'applicazione dello stimatore all'osservazione $y$ effettuata.

Si definisce l'\textit{errore di stima} $\mathrm{E}=X-\widehat{X}=X-g(Y)$ come la differenza tra la variabile da stimare e la sua stima. Un obiettivo naturale è quello di rendere tale errore di stima "piccolo" in accordo
a qualche criterio, deterministico o probabilistico, da precisare opportunamente. A tale
proposito, un criterio valido potrebbe essere quello di avere un errore di stima a media
nulla (\textit{stimatore non polarizzato}):
\[E[\mathrm{E}]=E[X-\widehat{X}]=0 \implies E[\widehat{X}]=E[X]\]
e di covarianza $E[\mathrm{E}\mathrm{E}^T]$ in qualche modo minima.

Come indice di prestazione per valutare la qualità di uno stimatore si introduce l’errore quadratico medio (MSE = Mean Square Error):
\begin{equation}
MSE_g(y)= E[(X-g(Y))^T(X-g(Y))|Y=y]
\end{equation}

Poiché uno stimatore con MSE inferiore è certamente da preferirsi, viene naturale porsi il problema della determinazione dello stimatore a minimo errore quadratico medio(stimatore MMSE = \textit{Minimum Mean Squared Error}) ovvero di uno stimatore $g^*(Y)$ il cui MSE sia inferiore a quello di ogni altro stimatore $g(Y)$.

\subsubsection{Stimatore MMSE}
Lo stimatore MMSE $g^*(\cdot)$ è dato da 
\begin{equation}
\label{stimatoremmse}
g^*(Y)=E[X|Y]
\end{equation}
e l’associato MMSE è dato da

\begin{equation}
\begin{split}
&E[(X-g^*(Y))^T(X-g^*(Y))|Y]\\
&=tr(var(X|Y)) \\
&=E[X^TX|Y]-E^T[X|Y]E[X|Y]
\end{split}
\end{equation}
\textit{Dimostrazione} - Lo stimatore MMSE deve minimizzare, rispetto a tutti gli stimatori $g^*(\cdot)$, il funzionale di costo\\
\begin{equation}
\begin{split}
V(g)&=E[(X-g(Y))^T(X-g(Y))|Y]= \\
&=E[X^TX+g^T(Y)g(Y)-X^Tg(Y)-g^T(Y)X|Y)]= \\
&=E[X^TX|Y]+g^T(Y)g(Y)-E^T[X|Y]g(Y)-g^T(Y)E[X|Y]=\\
&=\underbrace{(g(Y)-E[X|Y])^T(g(Y)-E[X|Y])}_{V_1(g)}+\underbrace{E[X^TX|Y]-E^T[X|Y]E[X|Y]}_{V_2}\\
\end{split}
\end{equation}
dove il primo termine $V_1(g)$, dipendente da $g^*(\cdot)$, può essere reso nullo (quindi minimo, essendo non-negativo) scegliendo $g(Y)=g^*(Y)$ uguale alla media condizionata come in \eqref{stimatoremmse}, mentre il secondo termine $V_2$, indipendente da $g^*(\cdot)$, coincide con la traccia della varianza condizionata $tr(var(X|Y))=tr(E[XX^T|Y]-E[X|Y]E^T[X|Y])=E[X^TX|Y]-E^T[X|Y]E[X|Y]\ge 0 $ e rappresenta il costo minimo (MMSE) $Vg^*(\cdot)$.\\

\subsubsection{Stima MMSE nel caso Gaussiano} 
Se $X$ e $Y$ sono congiuntamente Gaussiane (vedi \eqref{congiuntgauss}) allora lo stimatore MMSE è dato da
\begin{equation}
\begin{split}
\widehat{X}&=g^*(Y)=E[X|Y]=\\
&=\Sigma_{XY}\Sigma_Y^{-1}Y + \mu_{X}-\Sigma_{XY}\Sigma_Y^{-1}\mu_{Y}
\end{split}
\end{equation}
con relativo MMSE dato da:
\begin{equation}
tr(\widehat{\Sigma}_X)\triangleq tr(E[(X-g^*(Y))(X-g^*(Y))^T])=tr(\Sigma_X-\Sigma_{XY}\Sigma_Y^{-1}\Sigma_{XY}^T)
\end{equation}
$Dimostrazione $ - Nelle ipotesi fatte, sfruttando le formule della media e covarianza condizionata di variabili aleatorie Gaussiane, si ha
\begin{equation}
\begin{split}
\widehat{X}&=g^*(Y)=E[X|Y]=\\
&=\mu_{X}+ \Sigma_{XY}\Sigma_Y^{-1}(Y-y^{-1})=\\
&=\Sigma_{XY}\Sigma_Y^{-1}Y+[\mu_{X}-\Sigma_{XY}\Sigma_Y^{-1}\mu_{Y}]\\
\end{split}
\end{equation}
Inoltre,
\begin{equation}
\widehat{\Sigma}_X=var(X|Y)=\Sigma_X- \Sigma_{XY}\Sigma_Y^{-1}\Sigma_{XY}^T 
\end{equation}
come volevasi dimostrare.

Si può osservare come nel caso gaussiano lo stimatore MMSE sia a tutti gli effetti uno stimatore affine, ovvero della forma $g(Y)=AY+b$, dove:
\begin{align}
A&=\Sigma_{XY}\Sigma_Y^{-1}\\
b&=\mu_{X}-A\mu_{Y}=\mu_{X}-\Sigma_{XY}\Sigma_Y^{-1}\mu_{Y}
\end{align}

In generale lo stimatore MMSE non è affine, infatti esiste il problema della ricerca del miglior stimatore affine detto stimatore BLUE (\emph{Best Linear Unbiased Estimator} con una imprecisione terminologica essendo lo stimatore cercato affine e non lineare) che è meno preciso dello stimatore MMSE.\\
Nel caso gaussiano tuttavia essi coincidono, ovvero lo stimatore MMSE appena determinato è esso stesso lo stimatore BLUE per $X$ sulla base delle osservazioni di $Y$.
\newpage

\documentclass[12pt,a4paper]{article}
\usepackage[utf8]{inputenc}
\usepackage[italian]{babel}
\usepackage{amsmath}
\usepackage{amsfonts}
\usepackage{amssymb}
\title{Automatica}

\begin{document}

\section{Automatica}
\subsection{Sistemi dinamici a tempo discreto}

Un sistema dinamico a tempo discreto è il modello matematico di un oggetto che interagisce con l’ambiente circostante attraverso canali di ingresso e di uscita che sono rappresentati attraverso vettori, $\mathbf{u}$ e $\mathbf{y}$, di variabili dipendenti dal tempo. La differenza principale dai sistemi a tempo continuo è che in questo caso il tempo è rappresentato come una variabile intera $k$.

Si avrà pertanto che per ogni istante di tempo $k$ il sistema riceverà dei segnali in ingresso e risponderà con dei segnali in uscita.

Il vettore $\mathbf{u} \in \mathbb{R}^m$ rappresenta i segnali che l’oggetto riceve dall’esterno mentre il vettore $\mathbf{u} \in \mathbb{R}^p$ rappresenta i segnali che l’oggetto dà in uscita.


In generale il comportamento del sistema non dipende esclusivamente da questi due vettori, ovvero non vi è un legame diretto tra ingresso e uscita: infatti il sistema ha uno stato interno che evolve in funzione degli ingressi e degli stati precedenti. In particolare lo stato di un sistema può essere a sua volta rappresentato da un vettore $\mathbf{x} \in \mathbb{R}^n$.


Il modello del sistema è pertanto costituito da equazioni che descrivono l’evoluzione dello stato del sistema in funzione dell’ingresso, dello stato e del tempo e esprimono la relazione d'uscita:
\[
\begin{cases}
\mathbf{x}(k+1) = \mathbf{f}(\mathbf{x}(k),\mathbf{u}(k),k) \\
\mathbf{y}(k) = \mathbf{g}(\mathbf{x}(k),\mathbf{u}(k),k)
\end{cases}
\]
dove f e g sono opportune funzioni vettoriali.

Consideriamo una particolare tipologia di sistemi, quelli lineari strettamente propri, in cui le funzioni f e g sono appunto funzioni lineari e l’uscita non dipende esplicitamente dall’ingresso. In questo caso le equazioni si possono genericamente scrivere come:
\[
\begin{cases}
\mathbf{x}(k+1) = A(k)\mathbf{x}(k) + B(k)\mathbf{u}(k) \\
\mathbf{y}(k) = C(k)\mathbf{x}(k)
\end{cases}
\]
dove A,B,C sono matrici di coefficienti variabili nel tempo.

Tuttavia tali modelli sono approssimazioni ideali che possono essere valide in alcuni contesti, mentre in altri è necessario tener conto delle incertezze e delle imprecisioni che si hanno nella misura dei segnali di ingresso e di uscita del sistema.

Tali incertezze possono essere modellizzate come vettori casuali:
\[
\begin{cases}
x(k+1) = A(k)x(k) + B(k)u(k) + v(k)
y(k) = C(k)x(k) + w(k)
\end{cases}
\]
In particolare si possono fare alcune ipotesi su tali variabili casuali:

\end{document}

\section{Osservatore ottimo}
Nella teoria del controllo, l'osservatore è un sistema dinamico che ha lo scopo di stimare lo stato di un altro sistema. L'osservatore è utile in quanto la conoscenza dell'evoluzione dello stato di un processo permette di risolvere problemi come la stabilizzazione e il controllo.

L'osservatore più utilizzato nel caso di sistemi lineari prende il nome di \textit{Osservatore di Luenberger}\cite{bolzern} ed ha la seguente espressione:
\begin{equation}
\label{obsv}
\hat{x}_{k+1}=A_k\hat{x}_k+B_ku_k+K_k(y_k-C_k\hat{x}_k)
\end{equation}

Ad ogni istante di tempo, tale sistema calcola la nuova stima dello stato a partire dalla precedente sfruttando il modello noto del processo e le informazioni date dalle quantità conosciute $u_k$ e $y_k$, ovvero ingresso e uscita del processo. \\
In particolare il termine $A_k\hat{x}_k+B_ku_k$ altro non è che l'applicazione dell'equazione di aggiornamento dello stato \eqref{eqstate} alla stima calcolata nel passo precedente sfruttando la conoscenza dell'ingresso, mentre il termine $K_k(y_k-C_k\hat{x}_k)$ è la correzione che viene fatta sulla base della differenza tra l'uscita del processo $y_k$ e quella stimata dall'osservatore $C_k\hat{x}_k$.\\
Il fattore $K_k$ è il tassello essenziale per il corretto funzionamento dell'osservatore, dato che va a determinarne il comportamento (stabilità e velocità di convergenza).\\
In particolare, si può riscrivere l'equazione dell'osservatore nella forma:
\begin{equation}
\label{obsv2}
\hat{x}_{k+1}=(A_k-K_kC_k)\hat{x}_k+B_ku_k+K_ky_k
\end{equation}
Si osserva come la dinamica dell'osservatore sia determinata dalla matrice $A_k-K_kC_k$, pertanto la matrice $K_k$ va scelta in modo opportuno, ad esempio in modo tale che garantisca la stabilità del sistema, ovvero che $A_k-K_kC_k$ abbia autovalori $\lambda$ tali che $\lambda<1$ (nel caso di tempo discreto). Tale specifica può essere soddisfatta se e solo se il processo risulta \textit{osservabile}.\\
Nel caso di sistemi lineari stocastici, come si vedrà in seguito, il problema della determinazione della matrice $K_k$ è affrontato in modo da minimizzare l'errore quadratico medio di stima.\\
La scelta della stima $\hat{x}_0$ non è critica, dato che se l'osservatore è progettato correttamente gli effetti di una stima iniziale poco precisa si estinguono in un tempo relativamente breve grazie all'effetto del termine correttivo precedentemente descritto.

\subsection{Osservatore per sistemi stocastici}
Si considera il sistema lineare con disturbi di processo e misura \eqref{rumlinsys}, con stato iniziale $x_0$ e con ingresso $u_k$ misurabile per ogni $k \geq k_0$ e costruiamo l'osservatore come in \eqref{obsv}.

Per valutare la precisione dell'osservatore si definisce l'errore di stima, \\ $e_k \triangleq x_k-\hat{x}_k$, il quale è regolato da un'equazione dinamica che si può ottenere valutando tale espressione all'istante $k+1$ e sfruttando le equazioni \eqref{rumlinsys1} e \eqref{obsv}:
\begin{equation}
\label{errore}
\begin{split}
e_{k+1}&=A_kx_k+B_ku_k+W_kw_k-[A_k\hat{x}_k+B_ku_k+K_k(y_k-C\hat{x}_k)] = \\
&=(A_k-K_kC_k)e_k+W_kw_k-K_kv_k
\end{split}
\end{equation}

Tale errore, essendo definito sulla base dello stato $x_k$ del processo non è noto, tuttavia la sua definizione ci permetterà di caratterizzarlo dal punto di vista statistico, infatti osserviamo che l'errore è a sua volta un sistema stocastico, dato che la sua espressione dipende dai termini $v_k$ e $w_k$, pertanto ne definiamo la matrice di covarianza all'istante $k+1$:
\begin{equation}
\label{matrcov}
P_{k+1}=E[e_{k+1}e_{k+1}^T]
\end{equation}
La traccia di tale matrice rappresenta l'\textit{errore quadratico medio} di stima all'istante $k+1$.

%Osserviamo che il valore atteso dell'errore è un sistema autonomo:
%\begin{equation}
%\bar{e}_{k+1}=E[e_{k+1}]=(A_k-K_kC_k)\bar{e}_k+ E[W_kw_k] - E[K_kv_k]=(A_k-K_kC_k)\bar{e}_k
%\end{equation}

Per procedere con l'analisi ricordiamo le ipotesi \eqref{inizioipotesistatistiche}-\eqref{fineipotesistatistiche} fatte sui termini stocastici presenti nelle equazioni del sistema.
In particolare, sfruttiamo le ipotesi di incorrelazione tra i rumori e lo stato iniziale fatte in precedenza per dimostrare che anche l'errore di stima è incorrelato con i rumori.
Dalla teoria dei sistemi sappiamo che, risolvendo l'equazione alle differenze \eqref{errore} l'errore di stima $e_k$ è una combinazione lineare dell'errore iniziale $e_0=x_0-\hat{x}_0$ e dei rumori $w_i$ e $v_i$ per $i=0,1,...,k-1$:
\begin{equation}
e_k=\Phi_{k}e_0 + \sum_{i=0}^{k-1}\Phi_{k-i}W_iw_i - \sum_{j=0}^{k-1}\Phi_{k-j}K_jv_j
\end{equation}
dove $\Phi_{k}=\prod_{i=0}^{k-1}A_i-K_iC_i$.\\
Utilizzando tale espressione possiamo dimostrare l'incorrelazione tra l'errore di stima e i rumori di processo e misura.
\begin{equation}
\begin{split}
E[e_kw_k^T]&=E[(\Phi_{k}e_0 + \sum_{i=0}^{k-1}\Phi_{k-i}W_iw_i - \sum_{j=0}^{k-1}\Phi_{k-j}K_jv_j)w_k^T]=\\
&=\Phi_kE[e_0w_k^t] + \sum_{i=0}^{k-1}\Phi_{k-i}W_iE[w_iw_k^t] - \sum_{j=0}^{k-1}\Phi_{k-j}K_jE[v_jw_k^T]=\\
&=\Phi_kE[x_0w_k^t] - \Phi_k\hat{x}_0E[w_k^t] = 0
\end{split}
\end{equation}
Allo stesso modo si dimostra che anche $E[e_kv_k^T]=0$.
\newpage

\section{Ottimizzazione}
L'obiettivo che ci si prefigge è quello di determinare la matrice $\matr{L}_k$ tale che la stima fornita dall'osservatore sia il più attendibile possibile.
In particolare si vuole minimizzare l'errore quadratico medio di stima:
\[E[\vett{e}_k^T\vett{e}_k]=tr(\matr{P}_k)\]
Tale problema prende il nome di \textit{problema dell'osservatore ottimo}.
Se la matrice $\matr{R}_k$ è definita positiva $ \forall k \geq 0$ il problema si dice\textit{non singolare }.
Si dimostra \citep{kalmanbucy} che la soluzione del problema non singolare dell'osservatore ottimo è costituita da:
\[\matr{L}_k = \matr{P}_k\matr{C}_k^T(\matr{C}_k\matr{P}_k\matr{C}_k^T + \matr{R}_k)^{-1}\]
dove $\matr{P}_k$ è la soluzione dell'equazione di Riccati scritta nella forma:
\[\matr{P}_{k+1}=-\matr{A}_k\matr{P}_k\matr{C}_k^T[\matr{R}_k+\matr{C}_k\matr{P}_k\matr{C}_k^T]^{-1}\matr{C}_k\matr{P}_k\matr{A}_k^T+\matr{A}_k\matr{P}_k\matr{A}_k^T+\matr{Q}_k\]
scegliendo come stima iniziale $\vett{\hat{x}}_0=\vett{\bar{x}}_0$.

Il dispositivo così ottenuto è detto osservatore ottimo a tempo discreto; esso viene frequentemente indicato anche come\textit{filtro di Kalman}.

La matrice $\matr{L}_k$ è detta matrice di guadagno del filtro.   
\newpage

\section{Equazioni del filtro}

Il filtro di Kalman è un'implementazione ricorsiva degli algoritmi di stima che risolve il problema della ricostruzione dello stato di un sistema lineare.

Sia $\vett{\hat{x}}_k$ la stima $k$-esima dello stato $\vett{x}_k$ , e sia questa stima gaussiana a error medio nullo ($E[\vett{e}_k] = 0$) con covarianza $\matr{P}_k=E[\vett{e}_k\vett{e}_k^T]$.

Desiderando costruire una stima $\vett{\hat{x}}_{k+1}$, dovremo tenere conto di due sor-
genti di informazione, la conoscenza del modello (mediante l’utilizzo della
legge di propagazione dello stato) e la conoscenza delle misure. Distinguiamo
quindi due diverse stime di $\vett{x}_k$, una prima $\vett{\hat{x}}_k^-$ costruita conoscendo le misure sino $\vett{y}_{k-1}$, ed una seconda $\vett{\hat{x}}_k$, che utilizza anche la misura $\vett{y}_{k}$.

Basandoci sui risultati della sezione precedente, costruiamo quindi le stime:
\begin{align}
\vett{\hat{x}}_k^- &= \matr{A}_k\vett{\hat{x}}_{k-1} + \matr{B}_k\vett{u}_k \\
\matr{P}_k^- &= \matr{A}_k\matr{P}_{k-1}\matr{A}_k^T + \matr{B}^w_k\matr{Q}_k(\matr{B}^w_k)^T \\
\matr{L}_k &= \matr{P}_k^-\matr{C}_k^T(\matr{C}_k\matr{P}_k^-\matr{C}_k^T + \matr{R}_k)^{-1}\\
\vett{\hat{x}}_k &= \vett{\hat{x}}_k^- + \matr{L}_k(\vett{y}_k - \matr{C}_k\vett{\hat{x}}_k^-)\\
\matr{P}_k &= (\matr{I} - \matr{L}_k\matr{C}_k)\matr{P}_k^-
\end{align}

L’algoritmo così descritto prende il nome di filtro di Kalman discreto. \\ Il guadagno della innovazione nel filtro, $\matr{L}_k$ è fondamentalmente un rapporto tra la incertezza nella stima dello stato P e la incertezza nella misura R: conseguentemente, se le misure sono molto accurate (R piccola), la nuova stima $\vett{\hat{x}}_k$ sarà poco legata alla precedente. Se viceversa sono disponibili misure poco affidabili ma vecchie stime relativamente buone, si propagheranno queste nel futuro appoggiandosi sostanzialmente al modello del sistema.

\newpage

\section{Documentazione software MATLAB}
In questo paragrafo viene descritta l'implementazione del \textit{filtro di Kalman} come sistema dinamico ed il task implementato dal nostro gruppo in ambiente di programmazione \textit{MATLAB}, usando l'approccio della programmazione orientata agli oggetti.

\subsection{\texttt{sistema.m}}
\textit{MATLAB} presenta già un'implementazione dei modelli di sistemi dinamici lineari ma si è preferito realizzarne una nuova implementazione che considerasse anche gli errori di processo e di misura in modo da rispettare le ipotesi del problema.

In particolare nel file \texttt{sistema.m} viene implementata la \textit{classe} dei sistemi dinamici stocastici che utilizzeremo.

Per semplicità abbiamo implementato soltanto sistemi tempo invarianti, per cui tutte le matrici che definiscono il sistema sono costanti.
\subsubsection{Proprietà}
Le proprietà di cui dispongono gli oggetti di questa classe sono :
\begin{lstlisting}[frame=single]
classdef sistema < handle
%SISTEMA 
%Classe che descrive un sistema dinamico lineare tempo invariante

	properties (Access = protected)
		A,B,C,D,W,Q,R,x;    %A,B,C,D matrici del sistema
		% W matrice di guadagno del rumore di processo
		% Q matrice di covarianza del rumore di processo
		% R matrice di covarianza del rumore di misura
		n,m,p,q;      %n dim stato, m dim ingresso, p dim uscita, q dim rumore di processo
		u;       %ultimo ingresso ricevuto
	end
\end{lstlisting}
I metodi implementati sono il costruttore dell'oggetto che va ad inizializzarlo e le due equazioni di evoluzione dello stato interno e di uscita.\\
\newpage
\subsubsection{Costruttore}
La creazione dell'oggetto \texttt{sistema} avviene tramite l'inizializzazione dei suoi parametri:
\begin{lstlisting}[frame=single]
function obj = sistema(A,B,C,D,W,Q,R,x0)
\end{lstlisting}
Al costruttore vanno passate tutte le matrici relative al caso preso in analisi (comprese le covarianze) ed il suo stato iniziale.\\
Al suo interno vengono effettuati tutti i controlli necessari a verificare che i parametri rispettino le seguenti proprietà:
\begin{itemize}
\item $A$ deve essere quadrata (dimensione $n \times n$);
\item $B$ deve avere $n$ righe (dimensione $n \times m$);
\item $C$ deve avere $n$ colonne (dimensione $p \times n$);
\item $D$ deve essere di dimensione $p \times m$;
\item $W$ deve avere $n$ righe (dimensione $n \times q$);
\item $Q$ deve essere quadrata (dimensione $q \times q$) e definita positiva;
\item $R$ deve essere quadrata (dimensione $p \times p$) e definita positiva;
\item $x0$ vettore di lunghezza $n$.
\end{itemize}

\subsubsection{Evoluzione dello stato}
In \textit{MATLAB} nei metodi delle classi che utilizzano le proprietà delle stesse, risulta necessario passare come argomento l'oggetto corrente. Questo è possibile attraverso la parola chiave \texttt{obj}.
\begin{lstlisting}[frame=single]
function update(obj, u)
	% aggiorna lo stato del sistema
	if (nargin<2)
		u = zeros(obj.m,1);  % se u viene omesso si considera nullo
	end
	obj.u = u;  % salva l'ultimo ingresso ricevuto
	xn = obj.A*obj.x+obj.B*obj.u+obj.W*mvnrnd(zeros(obj.q,1),obj.Q)';
	% calcola il nuovo stato x(k+1) = Ax(k) + Bu(k) + Ww(k) : w = rumore di processo
	obj.x = xn;               % aggiorna lo stato con quello nuovo
end
\end{lstlisting}
La funzione accetta come parametro esterno l'ingresso dato al sistema; esso può essere omesso, in tal caso viene considerato nullo.\\
Implementa l'equazione di stato $x_{k+1}=Ax_k+Bu_k+Ww_k$ andando ad aggiornare la variabile di stato $x$ dell'oggetto.

\subsubsection{Lettura dell'uscita}
Il metodo \textit{leggiUscita} implementa l'equazione $y(t) = Cx(t)+Du(t)+w$ restituendo in output il valore di $y$.\\
Il metodo non necessita di ulteriori argomenti in ingresso:

\begin{lstlisting}[frame=single]
function y = leggiUscita(obj) % restituisce in output l'uscita del sistema 
\end{lstlisting}

In più è stata implementato il metodo per la lettura dello stato interno in quanto l'accesso diretto alle proprietà del sistema è, per ragioni di integrità, consentito unicamente all'oggetto stesso.
La lettura dello stato del sistema non sarebbe possibile nella realtà, infatti tale metodo viene utilizzato solo per monitorare il comportamento del sistema, tali dati non verranno utilizzati direttamente.
\begin{lstlisting}[frame=single]
function x = leggiStato(obj) % get dello stato per plot.
\end{lstlisting}

\newpage

\subsection{\texttt{filtrokalman.m}}
La classe \texttt{filtrokalman} è stata implementata come un estensione della precedente classe \texttt{sistema}, infatti il \textit{filtro di Kalman} essendo un osservatore dello stato è a sua volta un sistema dinamico.\\
Tale estensione si realizza attraverso il concetto di ereditarietà delle classi, infatti \texttt{kalmanfilter} eredita proprietà e metodi di \texttt{sistema} e ciò si indica attraverso il simbolo \texttt{<} :
\begin{lstlisting}[frame=single]
classdef kalmanfilter < sistema 
\end{lstlisting}
\subsubsection{Proprietà}
Oltre alle proprietà della classe \texttt{sistema} da essa ereditate, vengono introdotte la matrice di guadagno, la matrice di covarianza dello stato corretto e la predizione del prossimo stato e della relativa covarianza:
\begin{lstlisting}[frame=single]
properties %(Access = protected)
	L;          % matrice guadagno di Kalman
	P;          % matrice di covarianza dello stato corretto
	xPr, PPr;   % predizione dello stato e relativa covarianza
end
\end{lstlisting}
\subsubsection{Costruttore}
Come in \texttt{sistema.m} la classe \texttt{filtrokalman} accetta come argomenti in ingresso le matrici relative al modello del sistema da osservare e la stima iniziale dello stato (\texttt{x0}) con la relativa covarianza (\texttt{P0}); se quest'ultima è omessa viene considerata come valore di default la matrice identità di ordine $n$:
\begin{lstlisting}[frame=single]
function obj = kalmanfilter(A, B, C, D, Q, R, x0, P0)
\end{lstlisting}
All'interno del costruttore viene richiamato il costruttore della superclasse \texttt{sistema} al fine di inizializzare le variabili relative al modello nell'oggetto \texttt{filtrokalman}:
\begin{lstlisting}[frame=single, escapeinside={(*}{*)}]
	obj(*@*)sistema(A, B, C, D, Q, R, x0);
\end{lstlisting}
Inoltre, viene verificato che stima iniziale e covarianza siano valide.
\newpage
\subsubsection{Evoluzione}
Come per la superclasse corrispondente, la classe \textit{filtrokalman} ha un metodo per il calcolo dell'evoluzione del sistema. Il metodo ereditato dalla classe \texttt{sistema} viene sovrascritto (\textit{override}) in modo da implementare l'algoritmo ricorsivo di stima descritto nel capitolo precedente:
\begin{lstlisting}[frame=single]
function update(obj, u, y) % stima lo stato
	obj.u=u;
	
	%calcolo guadagno di Kalman
	obj.L = obj.PPr*obj.C'/(obj.C*obj.PPr*obj.C'+obj.R);
	
	%correzione
	obj.x = obj.xPr+obj.L*(y-obj.C*obj.xPr);
	I_LC = (eye(obj.n)-obj.L*obj.C);
	obj.P = I_LC*obj.PPr*I_LC'+obj.L*obj.R*obj.L';
	
	%predizione
	obj.xPr = obj.A*obj.x + obj.B*u;
	obj.PPr = obj.A*obj.P*obj.A'+obj.W*obj.Q*obj.W';
end
\end{lstlisting}
I parametri d'ingresso, oltre al riferimento all'oggetto, sono rispettivamente l'ingresso e l'uscita del sistema da osservare al tempo $k$.

\subsubsection{Lettura stima}
Una volta effettuato l'aggiornamento del filtro, per ottenere il valore della stima dello stato calcolata si utilizza il metodo \texttt{leggiStima} che restituisce il vettore dello stato stimato.
\begin{lstlisting}[frame=single]
function x = leggiStima(obj)
    x = obj.x;
end
\end{lstlisting}
Se si è interessati al filtraggio dell'uscita del sistema, si può utilizzare il metodo \texttt{leggiUscitaStimata} che restituisce l'uscita del sistema calcolata sulla base dello stato stimato.\\
Entrambi i metodi non necessitano di alcun parametro esterno.
\begin{lstlisting}[frame=single]
function y = leggiUscitaStimata(obj)
	y = obj.C*obj.x + obj.D*obj.u;
end
\end{lstlisting}
Sono stati implementati anche i metodi \texttt{leggiL} e \texttt{leggiP} che permettono di ottenere le matrici L e P del filtro all'istante corrente. Ciò sarà utile per osservarne l'evoluzione attraverso le funzioni grafiche di \textit{MATLAB}.
\begin{lstlisting}[frame=single]
function L = leggiL(obj)
	L = obj.L;
end
function P = leggiP(obj)
	P = obj.P;
end
\end{lstlisting}
\newpage

\subsection{Main task : \texttt{filtraggio.m}}
Il task che ci siamo prefissati di raggiungere è quello di ricostruire un segnale disturbato da rumore bianco gaussiano. Questa applicazione risulta molto utile in ambito ingegneristico in quanto anche i migliori trasduttori, per limiti costruttivi, presentano delle variazioni nelle misure seppur piccole.\\
Oltre a questo i trasduttori migliori sono reperibili soltanto ad un costo elevato, per cui si può pensare in certe condizioni di risparmiare sulla sensoristica applicando alle misure più rumorose di un eventuale trasduttore economico il \textit{filtro di Kalman} così da ottenere dei valori affidabili a prezzi più accessibili.

\begin{wrapfigure}[18]{R}{0.4\textwidth}
\centering
\includegraphics[width=0.38\textwidth]{mainfilterPOSS} 
\caption{Menu di scelta dei segnali}
\end{wrapfigure}
All'inizio dello script vengono definiti il tempo di campionamento e la durata della simulazione in secondi.\\
Viene poi visualizzato un menu che permette di scegliere la natura del segnale da filtrare; i segnali possibili sono tutti e soli quelli ottenibili come uscite da sistemi lineari.\\
I modelli dei generatori di segnale utilizzati vengono riportati nella sezione successiva.
Cliccando uno dei segnali il programma provvederà alla creazione del modello del generatore ed alla sua successiva discretizzazione attraverso le funzioni \textit{built-in} di \textit{MATLAB}.
\begin{lstlisting}[frame=single]
sys = ss(A,B,C,D);
sysd = c2d(sys,dt); 
[Ad,Bd,Cd,Dd] = ssdata(sysd);
\end{lstlisting}
Successivamente vengono definite le matrici di covarianza dei rumori di processo e misura. I valori di tali matrici possono essere variati per aumentare o diminuire la rumorosità del segnale da filtrare.
\begin{lstlisting}[frame=single]
Q=1e-3;
R=1e-1*eye(p);
\end{lstlisting}
Vengono a questo punto inizializzati gli oggetti relativi al generatore di segnale e al filtro di Kalman e si definisce il ciclo che esegue la simulazione.
\begin{lstlisting}[frame=single]
% inizializzazione sistema generatore di segnale rumoroso
sys=sistema(Ad,Bd,Cd,Dd,W,Q,R,x0);

% inizializzazione filtro di kalman
P0=eye(n);
k=filtrokalman(Ad,Bd,Cd,Dd,W,Q,R,zeros(n,1),P0);
\end{lstlisting}
\newpage
\begin{lstlisting}[frame=single]
%% simulazione
for i=1:length(t)
	x(:,i)=sys.leggiStato();     % lettura stato sistema (segnale da ricostruire, per plot)
	y(:,i)=sys.leggiUscita();    % lettura uscita sistema (segnale rumoroso)
	sys.update();                % calcolo del nuovo stato del sistema
	k.update(0,y(:,i));          % aggiornamento filtro --> parametri u=0 e y(segnale rumoroso)
	xs(:,i)=k.leggiStima();      % lettura stima kalman
	Lplot(:,:,i)=k.leggiL();     % lettura matrice L per animazione
	Pplot(:,:,i)=k.leggiP();     % lettura matrice P per animazione
end
\end{lstlisting}
Una volta completata la simulazione, viene visualizzato il plot con il confronto tra segnale da ricostruire, campioni rumorosi e segnale ricostruito dal filtro.
Nella stessa finestra viene visualizzata l'animazione delle matrici di covarianza dello stato e del guadagno.\\
Di seguito viene riportato un esempio di filtraggio di un segnale sinusoidale:

\begin{figure}[h]
	\centering
	\includegraphics[width=\textwidth]{esempioSinusoide}
\end{figure}

\newpage

\subsection{Modelli dei generatori di segnale}
I generatori di segnale sono sistemi lineari autonomi ($B=0$) ad uscita scalare ($p=1$). La loro evoluzione a partire dallo stato iniziale $x(0)$ genera segnali diversi in base alla natura della matrice dinamica $A$.\\
In particolare il segnale viene prodotto nell'ultima componente dello stato $x(t)$, quindi la matrice di uscita sarà del tipo: $C=\begin{pmatrix}0 & ... & 0 & 1\end{pmatrix}$.\\
Di seguito sono riportati i modelli utilizzati nell'applicazione, che per semplicità sono scritti a tempo continuo e poi discretizzati da MATLAB.
\begin{itemize}
	\item Segnali polinomiali di grado $n$:  
	\begin{equation*}
		x(t)=a_nt^n + ... + a_1t + a_0
	\end{equation*}
	\begin{equation*}
		A=\begin{pmatrix}
		0 & 1 & 0 & ... & 0\\
		0 & 0 & 1 & ... & 0\\
		\vdots & & & \ddots &\\
		0 & 0 & ... & 0 & 1\\
		0 & 0 & ... & 0 & 0
		\end{pmatrix} \in \mathbb{R}^{(n+1) \times (n+1)}, \qquad x(0)=\begin{pmatrix}a_0 \\ a_1 \\ ... \\ a_n\end{pmatrix}
	\end{equation*}
	\begin{itemize}
		\item Scalino  ($n=0$): $A=0$
		\item Rampa    ($n=1$): $A=\begin{pmatrix}0 & 1\\0 & 0\end{pmatrix}$
		\item Parabola ($n=2$): $A=\begin{pmatrix}0 & 1 & 0\\0 & 0 & 1\\0 & 0 & 0\end{pmatrix}$
	\end{itemize}
	\item Segnale esponenziale:
		\begin{equation*}
		x(t)=x_0e^{\alpha t}
		\end{equation*}
		\begin{equation*}
		A=\alpha \in \mathbb{R}, \qquad x(0)=x_0
		\end{equation*}
	\item Segnale sinusoidale ad ampiezza costante:
		\begin{equation*}
		x(t)=a\cos(\omega t)
		\end{equation*}
		\begin{equation*}
		A=\begin{pmatrix}0 & -\omega\\ \omega & 0\end{pmatrix} \in \mathbb{R}^{2 \times 2}, \qquad x(0)=\begin{pmatrix}a \\ 0\end{pmatrix}
		\end{equation*}
	\item Segnale sinusoidale smorzato:
	\begin{equation*}
	x(t)=ae^{\alpha t}\cos(\omega t)
	\end{equation*}
	\begin{equation*}
	A=\begin{pmatrix}\alpha & -\omega\\ \omega & \alpha\end{pmatrix} \in \mathbb{R}^{2 \times 2}, \qquad x(0)=\begin{pmatrix}a \\ 0\end{pmatrix}
	\end{equation*}
\end{itemize}

\newpage

\newpage
\nocite{conti}
\bibliographystyle{plain}
\bibliography{references}
\end{document}
