\documentclass{article}
\usepackage[utf8]{inputenc}
\usepackage[italian]{babel}
\usepackage{amsmath}
\usepackage{amsfonts}
\usepackage{amssymb}
\usepackage{natbib}
\usepackage{graphicx}
\usepackage{caption}
\captionsetup[figure]{labelformat=empty}

\newcommand*{\matr}{\mathbf}
\newcommand*{\vett}{\mathbf}

\title{Filtro di Kalman}
\author{Antonio Lanciotti, Lorenzo D'Agostino, Arment Pelivani}

\begin{document}

\maketitle

\begin{figure}[ht]
\centering
\includegraphics[scale=1]{Rudolf_Kalman.jpg} 
\caption{Rudolf E. Kalman}
\label{fig:kalman}
\end{figure}

\newpage

\tableofcontents

\newpage



\section{Introduzione}
Il filtro di Kalman è un osservatore ottimo dello stato per sistemi lineari in presenza di rumori gaussiani.

\newpage

\section{Cenni di probabilità}

Il filtro di Kalman è un algoritmo che mira alla ricostruzione dello stato interno di un sistema basandosi unicamente su una serie di misurazioni che, a causa di limiti costruttivi, sono soggette a rumore.

A causa della natura deterministica del problema, risulta necessario affrontare alcuni aspetti della teoria della probabilità, in particolare ci soffermeremo sul concetto di variabile aleatoria normale, o Gaussiana, con l'intento di fornire un modello matematico per gli errori di misura che siamo costretti ad affrontare.

\subsection{Variabili aleatorie o casuali.}

Una variabile casuale/aleatoria è una variabile che può assumere valori diversi in dipendenza da qualche fenomeno aleatorio.
In particolare diremo che una variabile casuale $X$ si dice continua se esiste una funzione $f(x)$ definita su tutto $\mathbb{R}$ : $P(X \in B) = \int_B f(x) dx$ dove la funzione $f$ si dice \textit{densità di probabilità} della variabile casuale $X$.

Una variabile aleatoria è quindi una variabile che può assumere valori appunto casuali la cui probabilità dipende dalla funzione di densità di probabilità ad essa associata. 

Le variabili casuali quindi risultano essere un valido strumento matematico per la modellazione dei rumori.

In particolare utilizzeremo le variabili casuali cosiddette Gaussiane o normali che sono caratterizzate dalla funzione densità di probabilità: \[f(x) = \frac{1}{{\sigma \sqrt {2\pi } }}e^{-\frac{1}{2}{(\frac{x-\mu}{\sigma})}^2}\] Per poterle descrivere a pieno dobbiamo però introdurre il concetto di media/valore atteso, di varianza e di covarianza.

\subsubsection{Valore atteso}

Nella teoria della probabilità il valore atteso di una variabile casuale $X$, è un numero indicato con $E[X]$ che formalizza l'idea di valore medio di un fenomeno aleatorio.
\[\ {\mathbb  {E}}[X]=\int _{{-\infty }}^{{\infty }}xf(x)dx\]

\noindent  Si noti che l'operatore valore atteso è lineare: \[E[aX + bY] = aE[X] + bE[Y]\]




\subsubsection{Varianza}

La varianza di una variabile aleatoria è una funzione, che fornisce una misura della variabilità dei valori assunti dalla variabile stessa; nello specifico, la misura di quanto essi si discostino quadraticamente rispettivamente dal valore atteso.

La varianza della variabile aleatoria $X$ è definita come il valore atteso del quadrato della variabile aleatoria centrata $X - E[X]$ :\[Var(X) = E[(X - E[X])^2]\]

\subsubsection{Covarianza}

In statistica e in teoria della probabilità, la covarianza di due variabili aleatorie è un numero che fornisce una misura di quanto le due varino assieme, ovvero della loro dipendenza.

La covarianza di due variabili aleatorie $X$ e $Y$ è il valore atteso dei prodotti delle loro distanze dalla media: \[Cov(X,Y)= E [(X-E[X])(Y-E[Y])]\]
Due variabili casuali si dicono \textit{incorrelate} se la loro covarianza è nulla.

La covarianza può essere considerata una generalizzazione della varianza \[Var(X) = Cov(X,X)\]

\subsection{Variabili gaussiane e modellazione dei rumori.}

Le variabili gaussiane sono particolari variabili aleatorie caratterizzate da due parametri, $\mu$ e $\sigma^2$, e sono indicate tradizionalmente con: \[X \sim N(\mu ,\sigma^2)\]
Si può dimostrare che per le variabili gaussiane vale che: \[E[X]= \mu \qquad Var[X]= \sigma^2\]

Come anticipato possiamo modellizzare i vettori di disturbo del sistema che consideriamo, attraverso l'utilizzo di variabili aleatorie gaussiane a media nulla e varianza $\sigma^2$, di dimensioni conformi a quelle del sistema considerato.

\newpage
\subsection{Vettori casuali}

Un vettore casuale è un vettore i cui elementi sono essi stessi variabili casuali.

Risulta necessario estendere le definizioni date in precedenza per caratterizzare rumori che agiscono su sistemi non scalari.

\subsubsection{Valore atteso}

Si dice valore atteso del vettore casuale $x \in \mathbb{R}^n$ il vettore dei valori attesi delle variabili casuali che lo compongono: \[E[\vett{x}] = \begin{pmatrix}
E[x_1] & E[x_2] & \dots & E[x_n]
\end{pmatrix}^T
\]
Si definisce inoltre il valore quadratico medio di $\vett{x}$ come $E[\vett{x}^T \vett{x}]$.

\subsubsection{Matrice di covarianza}

Si definisce matrice di covarianza del vettore casuale $\vett{x} \in \mathbb{R}^n$ la matrice $n \times n$: \[ Cov(\vett{x}, \vett{x}) = E[(\vett{x}-E[\vett{x}])(\vett{x}-E[\vett{x}])^T]\]
Per come è definita, la matrice di covarianza è una matrice simmetrica semidefinita positiva i cui elementi $\sigma^2_{ij}$ sono le covarianze tra gli elementi $x_i$ e $x_j$ del vettore $\vett{x}$.

A sua volta si definisce la matrice di \textit{cross-covarianza} tra due vettori casuali $\vett{x}$ e $\vett{y}$, la matrice \[ Cov(\vett{x}, \vett{y}) = E[(\vett{x}-E[\vett{x}])(\vett{y}-E[\vett{y}])^T]\]
Due vettori $\vett{x}$ e $\vett{y}$ si dicono \textit{incorrelati} se $Cov(\vett{x},\vett{y}) = 0$.

\newpage

\documentclass[12pt,a4paper]{article}
\usepackage[utf8]{inputenc}
\usepackage[italian]{babel}
\usepackage{amsmath}
\usepackage{amsfonts}
\usepackage{amssymb}
\title{Automatica}

\begin{document}

\section{Automatica}
\subsection{Sistemi dinamici a tempo discreto}

Un sistema dinamico a tempo discreto è il modello matematico di un oggetto che interagisce con l’ambiente circostante attraverso canali di ingresso e di uscita che sono rappresentati attraverso vettori, $\mathbf{u}$ e $\mathbf{y}$, di variabili dipendenti dal tempo. La differenza principale dai sistemi a tempo continuo è che in questo caso il tempo è rappresentato come una variabile intera $k$.

Si avrà pertanto che per ogni istante di tempo $k$ il sistema riceverà dei segnali in ingresso e risponderà con dei segnali in uscita.

Il vettore $\mathbf{u} \in \mathbb{R}^m$ rappresenta i segnali che l’oggetto riceve dall’esterno mentre il vettore $\mathbf{u} \in \mathbb{R}^p$ rappresenta i segnali che l’oggetto dà in uscita.


In generale il comportamento del sistema non dipende esclusivamente da questi due vettori, ovvero non vi è un legame diretto tra ingresso e uscita: infatti il sistema ha uno stato interno che evolve in funzione degli ingressi e degli stati precedenti. In particolare lo stato di un sistema può essere a sua volta rappresentato da un vettore $\mathbf{x} \in \mathbb{R}^n$.


Il modello del sistema è pertanto costituito da equazioni che descrivono l’evoluzione dello stato del sistema in funzione dell’ingresso, dello stato e del tempo e esprimono la relazione d'uscita:
\[
\begin{cases}
\mathbf{x}(k+1) = \mathbf{f}(\mathbf{x}(k),\mathbf{u}(k),k) \\
\mathbf{y}(k) = \mathbf{g}(\mathbf{x}(k),\mathbf{u}(k),k)
\end{cases}
\]
dove f e g sono opportune funzioni vettoriali.

Consideriamo una particolare tipologia di sistemi, quelli lineari strettamente propri, in cui le funzioni f e g sono appunto funzioni lineari e l’uscita non dipende esplicitamente dall’ingresso. In questo caso le equazioni si possono genericamente scrivere come:
\[
\begin{cases}
\mathbf{x}(k+1) = A(k)\mathbf{x}(k) + B(k)\mathbf{u}(k) \\
\mathbf{y}(k) = C(k)\mathbf{x}(k)
\end{cases}
\]
dove A,B,C sono matrici di coefficienti variabili nel tempo.

Tuttavia tali modelli sono approssimazioni ideali che possono essere valide in alcuni contesti, mentre in altri è necessario tener conto delle incertezze e delle imprecisioni che si hanno nella misura dei segnali di ingresso e di uscita del sistema.

Tali incertezze possono essere modellizzate come vettori casuali:
\[
\begin{cases}
x(k+1) = A(k)x(k) + B(k)u(k) + v(k)
y(k) = C(k)x(k) + w(k)
\end{cases}
\]
In particolare si possono fare alcune ipotesi su tali variabili casuali:

\end{document}

\newpage

\section{Osservatore ottimo}
Nella teoria del controllo, l'osservatore è un sistema dinamico che ha lo scopo di stimare lo stato di un altro sistema. L'osservatore è utile in quanto la conoscenza dell'evoluzione dello stato di un processo permette di risolvere problemi come la stabilizzazione e il controllo.

L'osservatore più utilizzato nel caso di sistemi lineari prende il nome di \textit{Osservatore di Luenberger}\cite{bolzern} ed ha la seguente espressione:
\begin{equation}
\label{obsv}
\hat{x}_{k+1}=A_k\hat{x}_k+B_ku_k+K_k(y_k-C_k\hat{x}_k)
\end{equation}

Ad ogni istante di tempo, tale sistema calcola la nuova stima dello stato a partire dalla precedente sfruttando il modello noto del processo e le informazioni date dalle quantità conosciute $u_k$ e $y_k$, ovvero ingresso e uscita del processo. \\
In particolare il termine $A_k\hat{x}_k+B_ku_k$ altro non è che l'applicazione dell'equazione di aggiornamento dello stato \eqref{eqstate} alla stima calcolata nel passo precedente sfruttando la conoscenza dell'ingresso, mentre il termine $K_k(y_k-C_k\hat{x}_k)$ è la correzione che viene fatta sulla base della differenza tra l'uscita del processo $y_k$ e quella stimata dall'osservatore $C_k\hat{x}_k$.\\
Il fattore $K_k$ è il tassello essenziale per il corretto funzionamento dell'osservatore, dato che va a determinarne il comportamento (stabilità e velocità di convergenza).\\
In particolare, si può riscrivere l'equazione dell'osservatore nella forma:
\begin{equation}
\label{obsv2}
\hat{x}_{k+1}=(A_k-K_kC_k)\hat{x}_k+B_ku_k+K_ky_k
\end{equation}
Si osserva come la dinamica dell'osservatore sia determinata dalla matrice $A_k-K_kC_k$, pertanto la matrice $K_k$ va scelta in modo opportuno, ad esempio in modo tale che garantisca la stabilità del sistema, ovvero che $A_k-K_kC_k$ abbia autovalori $\lambda$ tali che $\lambda<1$ (nel caso di tempo discreto). Tale specifica può essere soddisfatta se e solo se il processo risulta \textit{osservabile}.\\
Nel caso di sistemi lineari stocastici, come si vedrà in seguito, il problema della determinazione della matrice $K_k$ è affrontato in modo da minimizzare l'errore quadratico medio di stima.\\
La scelta della stima $\hat{x}_0$ non è critica, dato che se l'osservatore è progettato correttamente gli effetti di una stima iniziale poco precisa si estinguono in un tempo relativamente breve grazie all'effetto del termine correttivo precedentemente descritto.

\subsection{Osservatore per sistemi stocastici}
Si considera il sistema lineare con disturbi di processo e misura \eqref{rumlinsys}, con stato iniziale $x_0$ e con ingresso $u_k$ misurabile per ogni $k \geq k_0$ e costruiamo l'osservatore come in \eqref{obsv}.

Per valutare la precisione dell'osservatore si definisce l'errore di stima, \\ $e_k \triangleq x_k-\hat{x}_k$, il quale è regolato da un'equazione dinamica che si può ottenere valutando tale espressione all'istante $k+1$ e sfruttando le equazioni \eqref{rumlinsys1} e \eqref{obsv}:
\begin{equation}
\label{errore}
\begin{split}
e_{k+1}&=A_kx_k+B_ku_k+W_kw_k-[A_k\hat{x}_k+B_ku_k+K_k(y_k-C\hat{x}_k)] = \\
&=(A_k-K_kC_k)e_k+W_kw_k-K_kv_k
\end{split}
\end{equation}

Tale errore, essendo definito sulla base dello stato $x_k$ del processo non è noto, tuttavia la sua definizione ci permetterà di caratterizzarlo dal punto di vista statistico, infatti osserviamo che l'errore è a sua volta un sistema stocastico, dato che la sua espressione dipende dai termini $v_k$ e $w_k$, pertanto ne definiamo la matrice di covarianza all'istante $k+1$:
\begin{equation}
\label{matrcov}
P_{k+1}=E[e_{k+1}e_{k+1}^T]
\end{equation}
La traccia di tale matrice rappresenta l'\textit{errore quadratico medio} di stima all'istante $k+1$.

%Osserviamo che il valore atteso dell'errore è un sistema autonomo:
%\begin{equation}
%\bar{e}_{k+1}=E[e_{k+1}]=(A_k-K_kC_k)\bar{e}_k+ E[W_kw_k] - E[K_kv_k]=(A_k-K_kC_k)\bar{e}_k
%\end{equation}

Per procedere con l'analisi ricordiamo le ipotesi \eqref{inizioipotesistatistiche}-\eqref{fineipotesistatistiche} fatte sui termini stocastici presenti nelle equazioni del sistema.
In particolare, sfruttiamo le ipotesi di incorrelazione tra i rumori e lo stato iniziale fatte in precedenza per dimostrare che anche l'errore di stima è incorrelato con i rumori.
Dalla teoria dei sistemi sappiamo che, risolvendo l'equazione alle differenze \eqref{errore} l'errore di stima $e_k$ è una combinazione lineare dell'errore iniziale $e_0=x_0-\hat{x}_0$ e dei rumori $w_i$ e $v_i$ per $i=0,1,...,k-1$:
\begin{equation}
e_k=\Phi_{k}e_0 + \sum_{i=0}^{k-1}\Phi_{k-i}W_iw_i - \sum_{j=0}^{k-1}\Phi_{k-j}K_jv_j
\end{equation}
dove $\Phi_{k}=\prod_{i=0}^{k-1}A_i-K_iC_i$.\\
Utilizzando tale espressione possiamo dimostrare l'incorrelazione tra l'errore di stima e i rumori di processo e misura.
\begin{equation}
\begin{split}
E[e_kw_k^T]&=E[(\Phi_{k}e_0 + \sum_{i=0}^{k-1}\Phi_{k-i}W_iw_i - \sum_{j=0}^{k-1}\Phi_{k-j}K_jv_j)w_k^T]=\\
&=\Phi_kE[e_0w_k^t] + \sum_{i=0}^{k-1}\Phi_{k-i}W_iE[w_iw_k^t] - \sum_{j=0}^{k-1}\Phi_{k-j}K_jE[v_jw_k^T]=\\
&=\Phi_kE[x_0w_k^t] - \Phi_k\hat{x}_0E[w_k^t] = 0
\end{split}
\end{equation}
Allo stesso modo si dimostra che anche $E[e_kv_k^T]=0$.
\newpage

\newpage

\section{Equazioni del filtro}

Il filtro di Kalman è un'implementazione ricorsiva degli algoritmi di stima che risolve il problema della ricostruzione dello stato di un sistema lineare.

Sia $\vett{\hat{x}}_k$ la stima $k$-esima dello stato $\vett{x}_k$ , e sia questa stima gaussiana a error medio nullo ($E[\vett{e}_k] = 0$) con covarianza $\matr{P}_k=E[\vett{e}_k\vett{e}_k^T]$.

Desiderando costruire una stima $\vett{\hat{x}}_{k+1}$, dovremo tenere conto di due sor-
genti di informazione, la conoscenza del modello (mediante l’utilizzo della
legge di propagazione dello stato) e la conoscenza delle misure. Distinguiamo
quindi due diverse stime di $\vett{x}_k$, una prima $\vett{\hat{x}}_k^-$ costruita conoscendo le misure sino $\vett{y}_{k-1}$, ed una seconda $\vett{\hat{x}}_k$, che utilizza anche la misura $\vett{y}_{k}$.

Basandoci sui risultati della sezione precedente, costruiamo quindi le stime:
\begin{align}
\vett{\hat{x}}_k^- &= \matr{A}_k\vett{\hat{x}}_{k-1} + \matr{B}_k\vett{u}_k \\
\matr{P}_k^- &= \matr{A}_k\matr{P}_{k-1}\matr{A}_k^T + \matr{B}^w_k\matr{Q}_k(\matr{B}^w_k)^T \\
\matr{L}_k &= \matr{P}_k^-\matr{C}_k^T(\matr{C}_k\matr{P}_k^-\matr{C}_k^T + \matr{R}_k)^{-1}\\
\vett{\hat{x}}_k &= \vett{\hat{x}}_k^- + \matr{L}_k(\vett{y}_k - \matr{C}_k\vett{\hat{x}}_k^-)\\
\matr{P}_k &= (\matr{I} - \matr{L}_k\matr{C}_k)\matr{P}_k^-
\end{align}

L’algoritmo così descritto prende il nome di filtro di Kalman discreto. \\ Il guadagno della innovazione nel filtro, $\matr{L}_k$ è fondamentalmente un rapporto tra la incertezza nella stima dello stato P e la incertezza nella misura R: conseguentemente, se le misure sono molto accurate (R piccola), la nuova stima $\vett{\hat{x}}_k$ sarà poco legata alla precedente. Se viceversa sono disponibili misure poco affidabili ma vecchie stime relativamente buone, si propagheranno queste nel futuro appoggiandosi sostanzialmente al modello del sistema.

\newpage

\newpage

\section{Conclusione}
``A nonlinear differential equation of the Riccati type is derived for the covariance matrix of the optimal filtering error. The solution of this 'variance equation' completely specifies the optimal filter for either finite or infinite smoothing intervals and stationary or non-stationary statistics.
The variance equation is closely related to the Hamiltonian (canonical) differential equations of the calculus of variations. Analytic solutions are available in some cases. The significance of the variance equation is illustrated by examples which duplicate, simplify, or extend earlier results in this field.
The duality principle relating stochastic estimation and deterministic control problems plays an important role in the proof of theoretical results. In several examples, the estimation problem and its dual are discussed side-by-side.
Properties of the variance equation are of great interest in the theory of adaptive systems. Some aspects of this are considered briefly. '' \citep{kalmanbucy}

\bibliographystyle{plain}
\bibliography{references}
\end{document}
