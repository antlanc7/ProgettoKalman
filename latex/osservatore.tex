\section{Osservatore ottimo}
Nella teoria del controllo, l'osservatore è un sistema dinamico che ha lo scopo di stimare lo stato di un altro sistema. L'osservatore è utile in quanto la conoscenza dell'evoluzione dello stato di un processo permette di risolvere problemi come la stabilizzazione e il controllo di ogni genere, (dai \href {https://www.researchgate.net/publication/260398837_Industrial_Applications_of_the_Kalman_Filter_A_Review}{sistemi industriali} al \href{https://www.researchgate.net/publication/221908836_Kalman_Filter_Applications_for_Traffic_Management}{controllo del traffico}).\\

L'osservatore più utilizzato nel caso di sistemi lineari prende il nome di \textit{Osservatore di Luenberger}\cite{bolzern} ed ha la seguente espressione:
\begin{equation}
\label{obsv}
\hat{x}_{k+1}=A_k\hat{x}_k+B_ku_k+K_k(y_k-C_k\hat{x}_k)
\end{equation}
dove $\hat{x}_k$ rappresenta la stima dello stato $x_k$.\\
Ad ogni istante di tempo, tale sistema calcola la nuova stima dello stato a partire dalla precedente sfruttando il modello noto del processo e le informazioni date dalle quantità conosciute $u_k$ e $y_k$, ovvero ingresso e uscita del processo. \\
In particolare il termine $A_k\hat{x}_k+B_ku_k$ altro non è che l'applicazione dell'equazione di aggiornamento dello stato \eqref{eqstate} alla stima calcolata nel passo precedente sfruttando la conoscenza dell'ingresso, mentre il termine $K_k(y_k-C_k\hat{x}_k)$ è la correzione che viene fatta sulla base della differenza tra l'uscita del processo $y_k$ e quella stimata dall'osservatore $C_k\hat{x}_k$.\\
Il fattore $K_k$ è il tassello essenziale per il corretto funzionamento dell'osservatore, dato che va a determinarne il comportamento (stabilità e velocità di convergenza).\\
In particolare, si può riscrivere l'equazione dell'osservatore nella forma:
\begin{equation}
\label{obsv2}
\hat{x}_{k+1}=(A_k-K_kC_k)\hat{x}_k+B_ku_k+K_ky_k
\end{equation}
Si osserva come la dinamica dell'osservatore sia determinata dalla matrice $A_k-K_kC_k$, pertanto la matrice $K_k$ va scelta in modo opportuno, ad esempio in modo tale che garantisca la stabilità del sistema, ovvero che $A_k-K_kC_k$ abbia autovalori $\lambda$ tali che $\lambda<1$ (nel caso di tempo discreto). Tale specifica può essere soddisfatta se e solo se il processo risulta \textit{osservabile}.\\
Nel caso di sistemi lineari stocastici, come si vedrà in seguito, il problema della determinazione della matrice $K_k$ è affrontato in modo da minimizzare l'errore quadratico medio di stima.\\
La scelta della stima $\hat{x}_0$ non è critica, dato che se l'osservatore è progettato correttamente gli effetti di una stima iniziale poco precisa si estinguono in un tempo relativamente breve grazie all'effetto del termine correttivo precedentemente descritto.

\subsection{Osservatore per sistemi stocastici}
Si considera il sistema lineare con disturbi di processo e misura \eqref{rumlinsys}, con stato iniziale $x_0$ e con ingresso $u_k$ misurabile per ogni $k \geq 0$ e costruiamo l'osservatore come in \eqref{obsv}.

Per valutare la precisione dell'osservatore si definisce l'errore di stima, \\ $e_k \triangleq x_k-\hat{x}_k$, il quale è regolato da un'equazione dinamica che si può ottenere valutando tale espressione all'istante $k+1$ e sfruttando le equazioni \eqref{rumlinsys1} e \eqref{obsv}:
\begin{equation}
\label{errore}
\begin{split}
e_{k+1}&=A_kx_k+B_ku_k+W_kw_k-[A_k\hat{x}_k+B_ku_k+K_k(y_k-C\hat{x}_k)] = \\
&=(A_k-K_kC_k)e_k+W_kw_k-K_kv_k
\end{split}
\end{equation}
Tale errore, essendo definito sulla base dello stato $x_k$ del processo, non è noto, tuttavia la sua definizione ci permette di caratterizzarlo dal punto di vista statistico.

Utilizzando la stima dello stato prodotta dall'osservatore possiamo andare a costruire una stima della matrice di covarianza dello stato, considerando $E[x_k] \simeq \hat{x}_k$.
\begin{equation}
\label{matrcov}
\begin{split}
Cov(x_{k+1})&=E[(x_{k+1}-E[x_{k+1}])(x_{k+1}-E[x_{k+1}])^T] \simeq\\
&\simeq E[(x_{k+1}-\hat{x}_{k+1})(x_{k+1}-\hat{x}_{k+1})^T]=\\
&=E[e_{k+1}e_{k+1}^T] \triangleq P_{k+1}
\end{split}
\end{equation}
La traccia di tale matrice rappresenta l'\textit{errore quadratico medio} di stima all'istante $k+1$.

%Osserviamo che il valore atteso dell'errore è un sistema autonomo:
%\begin{equation}
%\bar{e}_{k+1}=E[e_{k+1}]=(A_k-K_kC_k)\bar{e}_k+ E[W_kw_k] - E[K_kv_k]=(A_k-K_kC_k)\bar{e}_k
%\end{equation}

Per procedere con l'analisi ricordiamo le ipotesi \eqref{inizioipotesistatistiche}-\eqref{fineipotesistatistiche} fatte sui termini stocastici presenti nelle equazioni del sistema.
In particolare, sfruttiamo le ipotesi di incorrelazione tra i rumori e lo stato iniziale fatte in precedenza per dimostrare che anche l'errore di stima è incorrelato con i rumori.
Dalla teoria dei sistemi sappiamo che, risolvendo l'equazione alle differenze \eqref{errore} l'errore di stima $e_k$ è una combinazione lineare dell'errore iniziale $e_0=x_0-\hat{x}_0$ e dei rumori $w_i$ e $v_i$ per $i=0,1,...,k-1$:
\begin{equation}
e_k=\Phi_{k}e_0 + \sum_{i=0}^{k-1}\Phi_{k-i}W_iw_i - \sum_{j=0}^{k-1}\Phi_{k-j}K_jv_j
\end{equation}
dove $\Phi_{k}=\prod_{i=0}^{k-1}(A_i-K_iC_i)$.\\
Utilizzando tale espressione possiamo dimostrare l'incorrelazione tra l'errore di stima e i rumori di processo e misura.
\begin{equation}
\begin{split}
E[e_kw_k^T]&=E[(\Phi_{k}e_0 + \sum_{i=0}^{k-1}\Phi_{k-i}W_iw_i - \sum_{j=0}^{k-1}\Phi_{k-j}K_jv_j)w_k^T]=\\
&=\Phi_kE[e_0w_k^T] + \sum_{i=0}^{k-1}\Phi_{k-i}W_iE[w_iw_k^T] - \sum_{j=0}^{k-1}\Phi_{k-j}K_jE[v_jw_k^T]=\\
&=\Phi_kE[x_0w_k^T] - \Phi_k\hat{x}_0E[w_k^T] = 0
\end{split}
\end{equation}
dove essendo $\hat{x}_0$ una quantità nota e non stocastica può essere portata fuori dall'operatore valore atteso.\\
Allo stesso modo si dimostra che anche $E[e_kv_k^T]=0$.
\newpage