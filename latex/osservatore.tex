\section{Osservatore ottimo}
Nella teoria del controllo, l'osservatore è un sistema dinamico che ha lo scopo di stimare lo stato di un altro sistema. L'osservatore è utile in quanto la conoscenza dell'evoluzione dello stato di un processo permette di risolvere problemi come la stabilizzazione e il controllo.

Nel caso di sistema lineare, l'osservatore può essere progettato come una copia del processo (del quale deve essere pertanto noto il modello) con l'aggiunta di un termine correttivo proporzionale alla differenza tra le uscite del processo e dell'osservatore. 

Tale osservatore prende il nome di \textit{Osservatore di Luenberger} ed ha la seguente espressione:
\begin{equation}
\label{obsv}
\hat{x}_{k+1}=A_k\hat{x}_k+B_ku_k+K_k(y_k-C\hat{x}_k)
\end{equation}

Si considera il sistema lineare con disturbi di processo e misura \eqref{rumlinsys},
con stato iniziale $x(k_0)=x_0$ e con ingresso $u_k$ misurabile per ogni $k \geq k_0$.

Per valutare la precisione dell'osservatore si definisce l'errore di stima, \\ $e_k \triangleq x_k-\hat{x}_k$, il quale è regolato da un'equazione dinamica che si può ottenere valutando tale espressione per $k=k+1$ e sostituendo sfruttando le equazioni \eqref{rumlinsys1} e \eqref{obsv}:
\begin{equation}
\label{errore}
\begin{split}
e_{k+1}&=A_kx_k+B_ku_k+W_kw_k-[A_k\hat{x}_k+B_ku_k+K_k(y_k-C\hat{x}_k)] = \\
&=(A_k-K_kC_k)e_k+W_kw_k-K_kv_k
\end{split}
\end{equation}
%Osserviamo che il valore atteso dell'errore è un sistema autonomo:
%\begin{equation}
%\bar{e}_{k+1}=E[e_{k+1}]=(A_k-K_kC_k)\bar{e}_k+ E[W_kw_k] - E[K_kv_k]=(A_k-K_kC_k)\bar{e}_k
%\end{equation}
Osserviamo che l'errore è a sua volta un sistema stocastico, dato che la sua espressione dipende dai termini $v_k$ e $w_k$, pertanto ne definiamo la matrice di covarianza all'istante $k+1$:
\begin{equation}
\label{matrcov}
P_{k+1}=E[e_{k+1}e_{k+1}^T]
\end{equation}
Tale matrice rappresenta l'\textit{errore quadratico medio} di stima all'istante $k+1$.

Per procedere con l'analisi ricordiamo le ipotesi \eqref{inizioipotesistatistiche}-\eqref{fineipotesistatistiche} fatte sui termini stocastici presenti nelle equazioni del sistema.