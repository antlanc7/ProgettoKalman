\section{Equazioni del filtro}

Il filtro di Kalman stima lo stato del processo in certi istanti di tempo e quindi realizza un feedback sulla base delle misure (rumorose).

Le equazioni del filtro di Kalman appartengono a due gruppi:
predizione dello stato e aggiornamento basato sulle misure.

Le equazioni di predizione dello stato proiettano in avanti lo stato corrente e la covarianza dell’errore di stima al fine di ottenere una stima a priori per il successivo istante temporale.
mentre le equazioni di aggiornamento dello stato realizzano il meccanismo in retroazione, cioè incorporano le nuove misure nella stima a priori per ottenere una stima a posteriori migliorata.

\begin{itemize}
\item[\textbf{Predizione:}]le equazioni proiettano lo stato e la covarianza dell’errore di stima in avanti dall’istante temporale $k-1$ all’istante $k$ 
\begin{align*}
\hat{x}_k^- &= A\hat{x}_{k-1} + Bu_k \\
P_k^- &= AP_{k-1}A^T + Q
\end{align*}

\item[\textbf{Aggiornamento:}] prima viene calcolata la matrice dei guadagni di Kalman $L_k$, quindi le misure $y_k$ sono usate per generare una stima dello stato a posteriori. Alla fine, viene calcolata una stima a posteriori
della covarianza dell’errore
\begin{align*}
L_k &= P_k^-C^T(CP_k^-C^T + R)^{-1}\\
\hat{x}_k &= \hat{x}_k^- + L_k(y_k - C\hat{x}_k^-)\\
P_k &= (I - L_kC)P_k^-
\end{align*}


\end{itemize}