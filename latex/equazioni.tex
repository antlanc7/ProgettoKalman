\section{Equazioni del filtro}

Il filtro di Kalman è un'implementazione ricorsiva degli algoritmi di stima che risolve il problema della ricostruzione dello stato di un sistema lineare.

Sia $\hat{x}_k$ la stima $k$-esima dello stato $x_k$ , e sia questa stima gaussiana a error medio nullo ($E[e_k] = 0$) con covarianza $P_k=E[e_ke_k^T]$.

Desiderando costruire una stima $\hat{x}_{k+1}$, dovremo tenere conto di due sor-
genti di informazione, la conoscenza del modello (mediante l’utilizzo della
legge di propagazione dello stato) e la conoscenza delle misure. Distinguiamo
quindi due diverse stime di $x_k$, una prima $\hat{x}_k^-$ costruita conoscendo le misure sino $y_{k-1}$, ed una seconda $\hat{x}_k$, che utilizza anche la misura $y_{k}$.

Basandoci sui risultati della sezione precedente, costruiamo quindi le stime:
\begin{align}
\hat{x}_k^- &= A_k\hat{x}_{k-1} + B_ku_k \\
P_k^- &= A_kP_{k-1}A_k^T + B^w_kQ_k(B^w_k)^T \\
L_k &= P_k^-C_k^T(C_kP_k^-C_k^T + R_k)^{-1}\\
\hat{x}_k &= \hat{x}_k^- + L_k(y_k - C_k\hat{x}_k^-)\\
P_k &= (I - L_kC_k)P_k^-
\end{align}

L’algoritmo così descritto prende il nome di filtro di Kalman discreto. \\ Il guadagno della innovazione nel filtro, $L_k$ è fondamentalmente un rapporto tra la incertezza nella stima dello stato P e la incertezza nella misura R: conseguentemente, se le misure sono molto accurate (R piccola), la nuova stima $\hat{x}_k$ sarà poco legata alla precedente. Se viceversa sono disponibili misure poco affidabili ma vecchie stime relativamente buone, si propagheranno queste nel futuro appoggiandosi sostanzialmente al modello del sistema.

\newpage