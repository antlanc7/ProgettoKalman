\section{Equazioni del filtro}

Il filtro di Kalman stima lo stato del processo in certi istanti di tempo e quindi realizza un feedback sulla base delle misure (rumorose).

Le equazioni del filtro di Kalman appartengono a due gruppi:
predizione dello stato e aggiornamento basato sulle misure.

Le equazioni di predizione dello stato proiettano in avanti lo stato corrente e la covarianza dell’errore di stima al fine di ottenere una stima a priori per il successivo istante temporale.
mentre le equazioni di aggiornamento dello stato realizzano il meccanismo in retroazione, cioè incorporano le nuove misure nella stima a priori per ottenere una stima a posteriori migliorata.

\begin{itemize}
\item[\textbf{Predizione:}]le equazioni proiettano lo stato e la covarianza dell’errore di stima in avanti dall’istante temporale $k-1$ all’istante $k$ 
\begin{align*}
\vett{\hat{x}}_k^- &= \matr{A}_k\vett{\hat{x}}_{k-1} + \matr{B}_k\vett{u}_k \\
\matr{P}_k^- &= \matr{A}_k\matr{P}_{k-1}\matr{A}_k^T + \matr{Q}_k
\end{align*}

\item[\textbf{Aggiornamento:}] prima viene calcolata la matrice dei guadagni di Kalman $L_k$, quindi le misure $y_k$ sono usate per generare una stima dello stato a posteriori. Alla fine, viene calcolata una stima a posteriori della covarianza dell’errore:
\begin{align*}
\matr{L}_k &= \matr{P}_k^-\matr{C}_k^T(\matr{C}_k\matr{P}_k^-\matr{C}_k^T + \matr{R}_k)^{-1}\\
\vett{\hat{x}}_k &= \vett{\hat{x}}_k^- + \matr{L}_k(\vett{y}_k - \matr{C}_k\vett{\hat{x}}_k^-)\\
\matr{P}_k &= (\matr{I} - \matr{L}_k\matr{C}_k)\matr{P}_k^-
\end{align*}

\end{itemize}
