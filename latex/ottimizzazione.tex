\subsection{Ottimizzazione}
L'obiettivo che ci si prefigge è quello di determinare la matrice $K_k$ tale che la stima fornita dall'osservatore sia il più attendibile possibile, ovvero ad ogni istante $k=0,1,...$ si vuole determinare ricorsivamente il guadagno $K_k$ dell' osservatore in modo da minimizzare l' errore quadratico medio di stima dello stato.

Sostituendo nella \eqref{matrcov} la \eqref{errore} e sfruttando le ipotesi \eqref{inizioipotesistatistiche}-\eqref{fineipotesistatistiche} si ottiene: 
\begin{equation}
\begin{split}
P_{k+1} &= E\left\{[(A_k - K_k C_k) e_k + W_k w_k - K_k v_k][(A_k -K_k C_k)e_k+W_k w_k-K_k v_k]^T\right\}=\\
 &= (A_k-K_kC_k)E[e_ke^T_k](A_k-K_kC_k)^T +W_kE[w_kw^T_k]W^T_k + K_kE[v_kv^T_k]K^T_k=\\
&= (A_k-K_kC_k)P_k(A_k-K_kC_k)^T + W_kQ_kW^T_k+K_kR_kW^T_k
\end{split}
\end{equation}

\noindent Il problema si riduce quindi alla seguente ottimizzazione quadratica: 
\begin{equation}
\begin{split}
K_k &= \underset{K}{\operatorname{argmin}} \left\{ \underbrace{(A_k-KC_k)P_k(A_k-KC_k)^T + W_kQ_kW^T_k+KR_kW^T_k}_{P_{k+1}} \right\} \\
&= \underset{K}{\operatorname{argmin}} \left\{ K\underbrace{(R_k+C_kP_kC_k^T)}_{S_k}K^T-K\underbrace{C_kP_kA_k^T}_{V^T_k}-\underbrace{A_kP_kC_k^T}_{V_k}K^T+W_kQ_kW_k^T+A_kP_kA_k^T \right\} \\
&=  \underset{K}{\operatorname{argmin}} \left\{KS_kK^T-KV^T_k-V_kK^T+A_kP_kA_k^T+W_kQ_kW_k^T\right\}
\end{split}
\end{equation}

\noindent Essendo $S_k \triangleq R_k+C_kP_kC_k^T > R_k > 0 $ la matrice $S_k$ risulta invertibile, pertanto si può scrivere: 
\begin{equation}
\begin{split}
K_k &= \underset{K}{\operatorname{argmin}} \left\{(K-V_kS^{-1}_k)S_k(K-V_kS^{-1}_k)^T - V_kS^{-1}_kV^T_k + A_kP_kA_k^T + W_kQ_kW_k^T\right\}\\ 
&= V_kS^{-1}_k\\
&= A_kP_kC_k^T(R_k+C_kP_kC_k)^{-1}
\end{split}
\end{equation}

\noindent L'ultima espressione fornisce quindi il guadagno ottimo $K_k$, all' istante $k$, a cui corrisponde il minimo errore quadratico medio all'istante $k+1$ dato da:
\begin{equation}
\begin{split}
P_{k+1}&=A_kP_kA_k^T-V_kS^{-1}_kV^T_k+W_kQ_kW_k^T\\ 
&=A_kP_kA_k^T-A_kP_kC_k^T(R_k+C_kP_kC_k^T)^{-1}C_kP_kA_k^T+W_kQ_kW_k^T
\end{split}
\end{equation}

\noindent In definitiva si ha il seguente risultato: \\
dato il sistema LTV \eqref{rumlinsys} che soddisfa le ipotesi \eqref{inizioipotesistatistiche}-\eqref{fineipotesistatistiche}, l'osservatore di Luenberger che minimizza ad ogni istante $k \geq 0$ l'errore quadratico medio di stima dello stato $P_k \triangleq E [(x_k - \hat{x}_k) (x_k - \hat{x}_k)T ]$ è fornito dal seguente algoritmo ricorsivo (che procede per $k = 0,1,...$):
\begin{align}
\label{luenb1}
K_k &= A_kP_kC_k^T(R_k+C_kP_kC_k)^{-1} \\
\label{luenb2}
\hat{x}_{k+1} &= A_k\hat{x}_k+B_ku_k+K_k(y_k-C\hat{x}_k)\\
\label{luenb3}
P_{k+1}&=A_kP_kA_k^T-A_kP_kC_k^T(R_k+C_kP_kC_k^T)^{-1}C_kP_kA_k^T+W_kQ_kW_k^T
\end{align}
\newpage