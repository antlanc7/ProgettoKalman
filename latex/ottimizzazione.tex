\section{Ottimizzazione}
L'obiettivo che ci si prefigge è quello di determinare la matrice $\matr{K}_k$ tale che la stima fornita dall'osservatore sia il più attendibile possibile.
In particolare si vuole minimizzare l'errore quadratico medio di stima:
\[E[\vett{e}_k^T\vett{e}_k]=tr(\matr{P}_k)\]
Tale problema prende il nome di \textit{problema dell'osservatore ottimo}.
\\A tal proposito consideriamo l' osservatore : \[\hat{x}_{k+1} = (\matr{A}-\matr{K}_k\matr{C})\hat{x}_k+b_k+\matr{K}_k y_k\]
La dinamica dell'errore di stima risulta quindi essere : \[\vett{e}_{k+1} = (\matr{A}_k-\matr{K}_k\matr{C})\vett{e}_k+\matr{W}\vett{w}_k-\matr{K}_k \vett{v}_k\]

Per definire correttamente il problema di ottimizzazione bisogna fare alcune ipotesi statistiche descritte nei paragrafi precedenti: 
\begin{itemize}
\item il disturbo di processo $\vett{w}_k$ ed il rumore di misura $\vett{v}_k$ sono rumori bianchi a media nulla e varianza rispettivamente $\matr{Q}_k , \matr{R}_k$;
\item lo stato iniziale $x_0$ è una variabile aleatoria di media $\hat{x}_0$ e varianza $\matr{P}_0$;
\item il disturbo di processo ed il rumore di misura sono mutuamente incorrelati : $E[\vett{w}_i \vett{v}^T_j] = 0$  $\forall$ $ i,j$;
\item il disturbo di processo ed il rumore di misura sono incorelati con lo stato iniziale.
\end{itemize}

A questo punto si pone il problema di \textit{sintesi dell;osservatore a minimo errore quadratico medio} : ad ogni istante $k=0,1,...$ si vuole determinare ricorsivamente il guadagno $\matr{K}_k$ dell' osservatore in modo da minimizzare
l' errore quadratico medio di stima dello stato all' istante $k+1$ : $\matr{P}_{k+1} \triangleq E[\vett{e}_{k+1}\vett{e}^T_{k+1}]$.
\\
Sostituendo all' interno dell' espressione di cui sopra l'espressione dell' errore di stima si ottiene : \[\matr{P}_{k+1} = E\left\{[(\matr{A}_k-\matr{K}_k\matr{C})\vett{e}_k+\matr{W}\vett{w}_{k}-\matr{K}_k \vett{v}_{k}][(\matr{A}_k-\matr{K}_k\matr{C})\vett{e}_k+\matr{W}\vett{w}_{k}-\matr{K}_k \vett{v}_{k}]^T\right\}\]
\\
Grazie alle ipotesi statistiche di cui sopra, si evince come l' errore di stima dello stato risulti essere incorrelato con il disturbo di processo ed il rumore di misura : \[E[\vett{e}_k \vett{w}^T_k] = 0, \,	E[\vett{e}_k \vett{v}^T_k] = 0\]\\
Sfruttando quindi le relazioni ottenute nella precedente espressione di $\matr{P}_{k+1}$ si ottiene quindi :
\begin{align}
\matr{P}_{k+1} &= (\matr{A}_k-\matr{K}_k\matr{C})E[\vett{e}_k\vett{e}^T_k](\matr{A}_k-\matr{K}_k\matr{C})^T +\matr{W}_kE[\vett{w}_k\vett{w}^T_k]\matr{W}^T_k + \matr{K}_kE[\vett{v}_k\vett{v}^T_k]\matr{K}^T_k \nonumber \\
&= (\matr{A}_k-\matr{K}_k\matr{C})\matr{P}_{k}(\matr{A}_k-\matr{K}_k\matr{C})^T + \matr{W}_k\matr{Q}_k\matr{W}^T_k+\matr{K}_k\matr{R}_k\matr{W}^T_k\nonumber
\end{align}
\\
Il problema si riduce quindi alla seguente ottimizzazione quadratica : 
\begin{align}
\matr{K}_k &= \underset{\matr{K}}{\operatorname{argmin}} \left\{ \underbrace{(\matr{A}-\matr{K}\matr{C})\matr{P}_{k}(\matr{A}-\matr{K}\matr{C})^T + \matr{W}\matr{Q}_k\matr{W}^T+\matr{K}\matr{R}_k\matr{W}^T}_{\matr{P}_{k+1}} \right\}\nonumber \\
&= \underset{\matr{K}}{\operatorname{argmin}} \left\{ \matr{K}\underbrace{(\matr{R}_k+\matr{C}\matr{P}_k\matr{C}^T)}_{\matr{S}_k}\matr{K}^T-\matr{K}\underbrace{\matr{C}\matr{P}_k\matr{A}^T}_{\matr{V}^T_k}-\underbrace{\matr{A}\matr{P}_k\matr{C}^T}_{\matr{V}_k}\matr{K}^T+\matr{W}\matr{Q}_k\matr{W}^T+\matr{A}\matr{P}_k\matr{A}^T \right\}\nonumber \\
&=  \underset{\matr{K}}{\operatorname{argmin}} \left\{\matr{K}\matr{S}_k\matr{K}^T-\matr{K}\matr{V}^T_k-\matr{V}_k\matr{K}^T+\matr{A}\matr{P}_k\matr{A}^T+\matr{W}\matr{Q}_k\matr{W}^T\right\}\nonumber
\end{align}
\\
Essendo $\matr{S}_k \triangleq \matr{R}_k+\matr{C}\matr{P}_k\matr{C}^T > \matr{R}_k > 0 $, per l' invertibilità di $\matr{R}_k$ ipotizzata, completando il quadrato sopra si ottiene : 
\begin{align}
\matr{K}_k &= \underset{\matr{K}}{\operatorname{argmin}} \left\{(\matr{K}-\matr{V}_k\matr{S}^{-1}_k)\matr{S}_k(\matr{K}-\matr{V}_k\matr{S}^{-1}_k)^T + \matr{A}\matr{P}_k\matr{A}^T+\matr{W}\matr{Q}_k\matr{W}^T-\matr{V}_k\matr{S}^{-1}_k\matr{V}^T_k\right\}\nonumber\\ 
&= \matr{V}_k\matr{S}^{-1}_k \nonumber\\
&= \matr{A}\matr{P}_k\matr{C}^T(\matr{R}_k+\matr{C}\matr{P}_k\matr{C})^{-1} \nonumber
\end{align}
\\
L'ultima espressione fornisce quindi il guadagno ottimo $\matr{K}_k$, all' istante $k$, a cui corrisponde il minimo errore quadratico medio all' istante $k+1$ dove :
\begin{align}
\matr{P}_{k+1}&=\matr{A}\matr{P}_k\matr{A}^T-\matr{V}_k\matr{S}^{-1}_k\matr{V}^T_k+\matr{W}\matr{Q}_k\matr{W}^T\nonumber\\ 
&=\matr{A}\matr{P}_k\matr{A}^T-\matr{A}\matr{P}_k\matr{C}^T(\matr{R}_k+\matr{C}\matr{P}_k\matr{C}^T)^{-1}\matr{C}\matr{P}_k\matr{A}^T+\matr{W}\matr{Q}_k\matr{W}^T \nonumber
\end{align}
\newpage